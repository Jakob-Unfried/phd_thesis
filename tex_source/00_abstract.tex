\chapter{Abstract}

Strongly correlated quantum many-body systems can exhibit a wide variety of phenomena, such as high-temperature superconductivity or the fractional quantum Hall effect.
%
Numerical simulations of these systems are challenging due to the exponential scaling of the dimension of the many-body Hilbert space with system size.
%
Tensor network methods offer powerful algorithms to approach these systems numerically, extracting predictions and measurable signatures from candidate theories for comparison with experiments.
%
We discuss several algorithmic improvements for tensor network algorithms.
%
In particular, we propose and benchmark (i) a modified truncation step in matrix product state simulations that enables the full benefit of hardware acceleration by avoiding singular value decompositions.
%
We propose and benchmark (ii) a gradient-based approach for the optimization of tensor networks for finite two-dimensional systems, enabling ground state search and time evolution.
%
Finally, we introduce (iii) a framework to incorporate nonabelian symmetries, as well as fermionic or anyonic exchange statistics into tensor network simulations on the tensor level.


\chapter{Zusammenfassung}

\begin{otherlanguage}{german}
    Stark korrelierte Quanten-Vielteilchensysteme können eine Vielzahl von interessanten Phänomenen aufweisen, wie z.B. Hochtemperatursupraleitung oder den fraktionalen Quanten-Hall-Effekt.
    %
    Die numerische Simulation solcher Systeme stellt jedoch eine Herausforderung dar, da die Dimension des Vielteilchen-Hilbertraums mit der Systemgröße exponentiell anwächst.
    %
    Tensornetzwerk-Methoden bieten leistungsstarke Algorithmen, um diese Systeme numerisch zu untersuchen, und aus Kanditaten einer Theorie des Systems Vorhersagen von Messgrößen zu extrahieren, und Vergleich mit Experimenten zu ermöglichen.
    %
    In dieser Arbeit stellen wir mehrere algorithmische Verbesserungen für Ten\-sor\-netz\-werk-Al\-go\-rith\-men vor.
    %
    Insbesondere (i) schlagen wir einen modifizierten Trunkierungsschritt in Ma\-trix\-pro\-dukt\-zu\-stand-Simulationen vor, der durch die Vermeidung von Singulärwertzerlegungen die volle Hardwarebeschleunigung, z.B.~von Grafikkarten ermöglicht, (ii) entwickeln und testen wir einen gradientenbasierten Ansatz zur Optimierung von Tensornetzwerken für endliche zweidimensionale Systeme, der sowohl die Simulation von Grundzuständen, als auch von Zeitentwicklung erlaubt, und (iii) stellen wir ein mathematischen Rahmen vor, um nichtabelsche Symmetrien sowie fermionische oder anyonische Austauschstatistiken auf der Tensorebene in Tensornetzwerk-Simulationen zu integrieren.
\end{otherlanguage}