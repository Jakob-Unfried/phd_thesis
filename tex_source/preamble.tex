\usepackage{subfiles}
\usepackage[utf8]{inputenc}
\usepackage[main=english,ngerman]{babel}
\usepackage{csquotes}
\usepackage{graphicx} % Required for inserting images
\usepackage[hidelinks]{hyperref}
\usepackage{bbm}  % for \mathbb and \mathbbm symbols
\usepackage{amsmath,amsfonts,amssymb}
\usepackage[chapter]{algorithm}  % for algorithm environment, option: use chapters for numering
\usepackage{enumitem}
\usepackage{mathtools}
\usepackage{paracol}  % independent columns in nonabelian section
\usepackage{tikz}
\usepackage{pgfplots}
\pgfplotsset{compat=1.18}
\usepackage{tikz-cd}  % commuting diagrams based on tikz
\usepackage{framed}
\usepackage[export]{adjustbox}
\usepackage{relsize}  % for \mathlarger command, e.g. for dagger of a whole diagram, want a large dagger.
\usepackage{mathdots}  % for \iddots in Z isomorphism matrix
\usepackage{booktabs}  % for clean tables (tabular environment) 
\usepackage{multirow}
\usepackage{caption}  % for setting caption width
\captionsetup{width=\linewidth}


\hyphenation{mo-noi-dal}
\hyphenation{di-men-sion-al}

% =============================================================================
% Colors
% =============================================================================


\definecolor{Green}{rgb}{0, 0.5, 0.2}
\ifcolors
    \newcommand{\red}[1]{{\color{red} {#1}}}
    \newcommand{\orange}[1]{{\color{orange} {#1}}}
    \newcommand{\blue}[1]{{\color{blue} {#1}}}
    \newcommand{\green}[1]{{\color{green} {#1}}}
    \definecolor{acrocolor}{rgb}{0, 0.3, 0.6}
    \definecolor{bg_left}{rgb}{0.95,0.92,0.9}
    \definecolor{bg_right}{rgb}{0.95,1.0,0.85}
\else
    \newcommand{\red}[1]{{#1}}
    \newcommand{\orange}[1]{{#1}}
    \newcommand{\blue}[1]{{#1}}
    \newcommand{\green}[1]{{#1}}
    \definecolor{bg_left}{rgb}{0.96,0.96,0.96}
    \definecolor{bg_right}{rgb}{0.93,0.93,0.93}
\fi

% monkeypatch \url command to give blue colors, eventhough we disabled coloring for all other hyperlinks.
\makeatletter
\let\old@url\url
\renewcommand{\url}[1]{\blue{\old@url{#1}}}
\makeatother

% =============================================================================
% Bibliography setup
% =============================================================================
\usepackage[citestyle=numeric-comp, bibstyle=ieee, sorting=none, minbibnames=2, maxbibnames=4, isbn=false]{biblatex}
\addbibresource{references.bib}
\AtEveryBibitem{
    \clearfield{issn}
    \clearfield{language}
    \clearfield{note}
}
\DeclareFieldFormat
  [article,inbook,incollection,inproceedings,patent,thesis,unpublished]
  {title}{\emph{#1\isdot}}
\DeclareFieldFormat{doi}{%
  \mkbibacro{DOI}\addcolon\space
  \ifhyperref
    {\href{https://doi.org/#1}{\red{\nolinkurl{#1}}}}
    {\nolinkurl{#1}}
}
% hack to use URL field only for select entries
\DeclareBibliographyCategory{needsurl}
\addtocategory{needsurl}{tenpySoftware}
\addtocategory{needsurl}{tensorkit-docs}
\renewbibmacro*{url+urldate}{%
  \ifcategory{needsurl}
    {\mkbibacro{DOI}\addcolon\space: \href{\thefield{url}}{\red{\thefield{url}}}%
     }
    {}}


% =============================================================================
% Terminology and Jargon Sectopn
% =============================================================================
\newlist{jargon}{description}{1}
\setlist[jargon]{leftmargin=0.2\hsize,labelindent=0em,itemindent=!,labelwidth=!}

% =============================================================================
% Definition , Theorem , ...
% =============================================================================
\newenvironment{proof}{%
    \color{aurometalsaurus}%
    \emph{Proof:}~%
}{
    \hfill$\square$
}

\newtheorem{definition}{Definition}[section]
\newtheorem{theorem}[definition]{Theorem}
\newtheorem{corollary}[definition]{Corollary}
\definecolor{aurometalsaurus}{rgb}{0.43, 0.5, 0.5}
\AtBeginEnvironment{proof}{\color{aurometalsaurus}}
% \newenvironment{proof}{%
%     \color{gray}
%     Proof: 
% }{%
% }

% =============================================================================
% ACRONYMS
% =============================================================================
\usepackage[acronym]{glossaries}
\makeglossaries
\setacronymstyle{long-short}

\ifcolors
    \newcommand{\acro}[1]{{\color{acrocolor} \gls{#1}}}
    \newcommand{\acrop}[1]{{\color{acrocolor} \glspl{#1}}}
    \newcommand{\acroshort}[1]{{\color{acrocolor} \acrshort{#1}}}
    \newcommand{\acrofull}[1]{{\color{acrocolor} \acrfull{#1}}}
    \newcommand{\acropfull}[1]{{\color{acrocolor} \acrfull{#1}}}
\else
    \newcommand{\acro}[1]{{\gls{#1}}}
    \newcommand{\acrop}[1]{{\glspl{#1}}}
    \newcommand{\acroshort}[1]{{\acrshort{#1}}}
    \newcommand{\acrofull}[1]{{\acrfull{#1}}}
    \newcommand{\acropfull}[1]{{\acrfull{#1}}}
\fi

% \newacronym{1d}{1D}{one-dimensional}
% \newacronym{2d}{2D}{two-dimensional}
\newacronym{obc}{OBC}{open boundary conditions}
\newacronym{pbc}{PBC}{periodic boundary conditions}
\newacronym{mps}{MPS}{matrix product state}
\newacronym{imps}{iMPS}{infinite matrix product state}
\newacronym{fcs}{FCS}{finitely correlated state}
\newacronym{mpo}{MPO}{matrix product operator}
\newacronym{ttn}{TTN}{tree tensor network}
\newacronym{mera}{MERA}{multiscale entanglement renormalization ansatz}
\newacronym{peps}{PEPS}{projected entangled pair state}
\newacronym{ipeps}{iPEPS}{infinite projected entangled pair state}
\newacronym{trg}{TRG}{tensor renormalization group}
\newacronym{tnr}{TNR}{tensor network renormalization}
\newacronym{ctmrg}{CTMRG}{corner transfer renormalization group}
\newacronym{fu}{FU}{full update}
\newacronym{ffu}{FFU}{fast full update}
\newacronym{su}{SU}{simple update}
\newacronym{bmps}{bMPS}{boundary matrix product state}
\newacronym{bmpo}{bMPO}{bulk matrix product operator}
\newacronym{pepo}{PEPO}{projected entangled pair operator}
\newacronym{tps}{TPS}{tensor product state}
\newacronym{tns}{TNS}{tensor network state}
\newacronym{dmrg}{DMRG}{density matrix renormalization group}
\newacronym{idmrg}{iDMRG}{infinite density matrix renormalization group}
\newacronym{tebd}{TEBD}{time evolving block decimation}
\newacronym{itebd}{iTEBD}{infinite time evolving block decimation}
\newacronym{tdvp}{TDVP}{time dependent variational principle}
\newacronym{ed}{ED}{exact diagonalization}
\newacronym{svd}{SVD}{singular value decomposition}
\newacronym{rsvd}{rSVD}{randomized singular value decomposition}
\newacronym{dsvd}{dSVD}{deformed singular value decomposition}
\newacronym{tqr}{tQR}{truncated QR-like decomposition}
\newacronym{qr}{QR}{QR decomposition}
\newacronym{qlp}{QLP}{QLP decomposition}
\newacronym{lq}{LQ}{LQ decomposition}
\newacronym{qrcp}{QRCP}{QR decomposition with column pivoting}
\newacronym{tenpy}{TeNPy}{Tensor Network Python}
\newacronym{fft}{FFT}{fast Fourier transform}
\newacronym{srft}{SRFT}{subsampled random Fourier transform}
\newacronym{vumps}{VUMPS}{variational uniform matrix product state}
\newacronym{mpoEvolution}{MPO evolution}{matrix operator based time evolution}
\newacronym{cpu}{CPU}{central processing unit}
\newacronym{gpu}{GPU}{graphics processing unit}
\newacronym{tpu}{TPU}{tensor processing unit}
\newacronym{ai}{AI}{artificial intelligence}
\newacronym{autodiff}{AD}{automatic differentiation}
\newacronym{flop}{FLOP}{floating point operation}
\newacronym{lbfgs}{L-BFGS}{Limited memory Broyden--Fletcher--Goldfarb--Shanno}
\newacronym{tfim}{TFIM}{transverse field Ising model}
\newacronym{dsf}{DSF}{dynamical spin structure factor}
\newacronym{als}{ALS}{alternating least squares}

% =============================================================================
% TYPESETTING ALGORITHMS
% =============================================================================
\newenvironment{Algorithm}[2]{%
    \begin{algorithm}
    \caption{#1}
    \vspace{1ex}
    #2
    \vspace{1ex}
    \begin{enumerate}[font=\footnotesize]
    \let\step\item
    \newenvironment{substeps}{%
        \begin{enumerate}[font=\footnotesize,label*=\alph*.]
        \let\substep\item
    }{%
        \end{enumerate}
    }
}{%
    \end{enumerate}
    \vspace{1ex}
    \end{algorithm}
}
% Usage:
% \begin{Algorithm}{Title}{Description of inputs, outputs, purpose (can be many lines)}
%     \step Do a thing
%     \step Do another thing
% \end{Algorithm}

% 

% =============================================================================
% TIKZ for monoidal categories
% =============================================================================
\usetikzlibrary{decorations.markings, positioning, calc, angles, ext.positioning-plus, shapes.misc, shapes.geometric, arrows.meta, pgfplots.fillbetween, backgrounds, babel}

%settings
\newcommand\arrowstyle{stealth}
\newcommand\arrowlinewidth{1.5pt}
\newcommand\arrowshift{3pt}  % through trial and arrow. depends on arrow head and line style!

% from https://tex.stackexchange.com/questions/264589/string-diagrams-in-monoidal-categories
\tikzset{
  tensor/.style={
    rounded rectangle, draw, align=center, minimum width=.7cm, minimum height=.7cm
  },
  mpo tensor/.style={
    chamfered rectangle, draw, align=center, minimum width=.7cm, minimum height=.7cm
  },
  left iso/.style={
    rounded rectangle, draw, align=center, minimum width=.7cm, minimum height=.7cm, rounded rectangle left arc=none
  },
  right iso/.style={
    rounded rectangle, draw, align=center, minimum width=.7cm, minimum height=.7cm, rounded rectangle right arc=none
  },
  morphism/.style={
    chamfered rectangle,draw,chamfered rectangle corners=north west,fill=black!15,minimum width=1cm,minimum height=1cm,align=center,inner xsep=7pt,
  },
  daggered/.style={
    chamfered rectangle,draw,chamfered rectangle corners=south west,fill=black!15,minimum width=1cm,minimum height=1cm,align=center,inner xsep=7pt,
  },
  transposed/.style={
    chamfered rectangle,draw,chamfered rectangle corners=south east,fill=black!15,minimum width=1cm,minimum height=1cm,align=center,inner xsep=7pt,
  },
  daggertransposed/.style={
    chamfered rectangle,draw,chamfered rectangle corners=north east,fill=black!15,minimum width=1cm,minimum height=1cm,align=center,inner xsep=7pt,
  },
  old z iso/.style={
    minimum width=.5cm,minimum height=.5cm,
    path picture={
      \draw[draw]
        let \p{diag}=($(path picture bounding box.north east)-(path picture bounding box.south west)$) in
        (path picture bounding box.south west) -- ++(0,0.4*\y{diag})
        arc [start angle=180, end angle=90, x radius=0.6*\x{diag}, y radius=0.6*\y{diag}]
        -- (path picture bounding box.north east) -- (path picture bounding box.south east) -- cycle;
    }
  },
  old z iso daggered/.style={
    minimum width=.5cm,minimum height=.5cm,
    path picture={
      \draw[draw]
        let \p{diag}=($(path picture bounding box.north east)-(path picture bounding box.south west)$) in
        (path picture bounding box.north west) -- ++(0,-0.4*\y{diag})
        arc [start angle=180, end angle=270, x radius=0.6*\x{diag}, y radius=0.6*\y{diag}]
        -- (path picture bounding box.south east) -- (path picture bounding box.north east) -- cycle;
    }
  },
  old z iso transposed/.style={
    minimum width=.5cm,minimum height=.5cm,
    path picture={
      \draw[draw]
        let \p{diag}=($(path picture bounding box.north east)-(path picture bounding box.south west)$) in
        (path picture bounding box.south west) -- ++(0.4*\x{diag},0)
        arc [start angle=270, end angle=360, x radius=0.6*\x{diag}, y radius=0.6*\y{diag}]
        -- (path picture bounding box.north east) -- (path picture bounding box.north west) -- cycle;
    }
  },
  z iso/.style={
    minimum width=.5cm,minimum height=.5cm,
    append after command={
        \pgfextra
            \draw[draw]
                let \p{diag}=($(\tikzlastnode.north east)-(\tikzlastnode.south west)$) in
                (\tikzlastnode.south west)
                -- ++(0,0.4*\y{diag})
                arc [start angle=180, end angle=90, x radius=0.6*\x{diag}, y radius=0.6*\y{diag}]
                -- (\tikzlastnode.north east)
                -- (\tikzlastnode.south east)
                -- cycle;
            ;
        \endpgfextra
    },
  },
  z iso daggered/.style={
    minimum width=.5cm,minimum height=.5cm,
    append after command={
        \pgfextra
            \draw[draw]
                let \p{diag}=($(\tikzlastnode.north east)-(\tikzlastnode.south west)$) in
                (\tikzlastnode.north west)
                -- ++(0,-0.4*\y{diag})
                arc [start angle=180, end angle=270, x radius=0.6*\x{diag}, y radius=0.6*\y{diag}]
                -- (\tikzlastnode.south east)
                -- (\tikzlastnode.north east)
                -- cycle;
            ;
        \endpgfextra
    },
  },
  z iso transposed/.style={
    minimum width=.5cm,minimum height=.5cm,
    append after command={
        \pgfextra
            \draw[draw]
                let \p{diag}=($(\tikzlastnode.north east)-(\tikzlastnode.south west)$) in
                (\tikzlastnode.south west)
                -- ++(0.4*\x{diag},0)
                arc [start angle=270, end angle=360, x radius=0.6*\x{diag}, y radius=0.6*\y{diag}]
                -- (\tikzlastnode.north east)
                -- (\tikzlastnode.north west)
                -- cycle;
            ;
        \endpgfextra
    },
  },
  old x tensor/.style={
    % note: do *not* include draw here. we do not want to draw the bounding box.
    minimum width=1cm,minimum height=1cm,align=center,
    path picture={
      \draw[draw,fill=black!15] let \p{diag}=($(path picture bounding box.north east)-(path picture bounding box.south west)$) in
        (path picture bounding box.south west)
        -- (path picture bounding box.west)
        arc [start angle=180, end angle=0, x radius=0.5*\x{diag}, y radius=0.5*\y{diag}]
        -- (path picture bounding box.south east)
        -- cycle;
    }
  },
  x tensor/.style={
    minimum width=1cm,minimum height=1cm,align=center,
    append after command={
        \pgfextra
            \draw[draw,fill=black!15]
                let \p{diag}=($(\tikzlastnode.north east)-(\tikzlastnode.south west)$) in
                (\tikzlastnode.south west)
                -- (\tikzlastnode.west)
                arc [start angle=180, end angle=0, x radius=0.5*\x{diag}, y radius=0.5*\y{diag}]
                -- (\tikzlastnode.south east)
                -- cycle
            ;
        \endpgfextra
    },
  },
  old y tensor/.style={
    % note: do *not* include draw here. we do not want to draw the bounding box.
    minimum width=1cm,minimum height=1cm,align=center,
    path picture={
      \draw[draw,fill=black!15] let \p{diag}=($(path picture bounding box.north east)-(path picture bounding box.south west)$) in
        (path picture bounding box.north west)
        -- (path picture bounding box.west)
        arc [start angle=180, end angle=360, x radius=0.5*\x{diag}, y radius=0.5*\y{diag}]
        -- (path picture bounding box.north east)
        -- cycle;
    }
  },
  y tensor/.style={
    minimum width=1cm,minimum height=1cm,align=center,
    append after command={
        \pgfextra
            \draw[draw,fill=black!15]
                let \p{diag}=($(\tikzlastnode.north east)-(\tikzlastnode.south west)$) in
                (\tikzlastnode.north west)
                -- (\tikzlastnode.west)
                arc [start angle=180, end angle=360, x radius=0.5*\x{diag}, y radius=0.5*\y{diag}]
                -- (\tikzlastnode.north east)
                -- cycle
            ;
        \endpgfextra
    },
  },
  peps tensor/.style={circle, draw, fill=green!50},
  pepo tensor/.style={diamond, draw, fill=red!50},
  double layer tensor/.style={rectangle, draw, fill=blue!40},
  inclusion/.style={
    kite,kite vertex angles=165 and 60,draw,fill=black!15,minimum height=1cm,align=center,inner xsep=0
  },
  projection/.style={
    kite,kite vertex angles=60 and 165,draw,fill=black!15,minimum height=1cm,align=center,inner xsep=0
  },
  space/.style={
    minimum height=1cm,align=center
  },
  arrow1/.style n args={3}{% e.g. \draw[arrow1={0.5}{left}{$A$}], latter two may be empty
    decoration={
      markings,
      mark=at position {#1*\pgfdecoratedpathlength+\arrowshift} with {\arrow[line width=\arrowlinewidth]{\arrowstyle}},
      mark=at position #1 with {\node[#2] {#3};},
    },
    postaction=decorate  
  },
  arrow1/.default={0.5}{}{},
  arrow1rev/.style n args={3}{
    decoration={
      markings,
      mark=at position {#1*\pgfdecoratedpathlength-\arrowshift} with {\arrowreversed[line width=\arrowlinewidth]{\arrowstyle}},
      mark=at position #1 with {\node[#2] {#3};},
    },
    postaction=decorate  
  },
  arrow1rev/.default={0.5}{}{},
  arrow2/.style n args={6}{% e.g. \draw[arrow2={0.2}{left}{$A$}{0.8}{left}{$A$}]
    decoration={
      markings,
      mark=at position {#1*\pgfdecoratedpathlength+\arrowshift} with {\arrow[line width=\arrowlinewidth]{\arrowstyle};}, 
      mark=at position #1 with {\node[#2] {#3};},
      mark=at position {#4*\pgfdecoratedpathlength+\arrowshift} with {\arrow[line width=\arrowlinewidth]{\arrowstyle};},
      mark=at position #4 with {\node[#5] {#6};},
    },
    postaction=decorate  
  },
  arrow2/.default={0.5}{}{}{0.5}{}{},
  arrow2rev/.style n args={6}{
    decoration={
      markings,
      mark=at position {#1*\pgfdecoratedpathlength-\arrowshift} with {\arrowreversed[line width=\arrowlinewidth]{stealth};},
      mark=at position #1 with {\node[#2] {#3};},
      mark=at position {#4*\pgfdecoratedpathlength-\arrowshift} with {\arrowreversed[line width=\arrowlinewidth]{stealth};},
      mark=at position #4 with {\node[#5] {#6};},
    },
    postaction=decorate  
  },
  arrow2rev/.default={0.5}{}{}{0.5}{}{},
  overdraw/.style={preaction={draw,white,line width=#1}},
  overdraw/.default=8pt,
}

\newcommand\overbraid{}  % just for safety, check if its not yet defined
\def\overbraid(#1)(#2)(#3)(#4){%
    % Usage: \overbraid(bottomleft)(bottomright)(topleft)(topright)
    \draw (#2) to [out=90, in=270] (#3);
    \draw[overdraw] (#1) to [out=90, in=270] (#4);
}
\newcommand\underbraid{}  % just for safety, check if its not yet defined
\def\underbraid(#1)(#2)(#3)(#4){%
    % Usage: \underbraid(bottomleft)(bottomright)(topleft)(topright)
    \draw (#1) to [out=90, in=270] (#4);
    \draw[overdraw] (#2) to [out=90, in=270] (#3);
}

\newcommand\rightovertwist{}
\def\rightovertwist(#1)[#2]{%
    % usage: \rightovertwist(bottomCoord)[nameForTopCoord]
    \coordinate[above=10pt of #1] (#2);
    \draw (#2)
        to[out=270,in=180] ++(8pt,-10pt)
        to[out=0,in=270] ++(5pt,5pt)
        coordinate (twist);
    \draw[overdraw=4pt] (twist) to[out=90,in=0] ++(-5pt, 5pt) to[out=180,in=90] ++(-8pt, -10pt);
}
\newcommand\rightundertwist{}
\def\rightundertwist(#1)[#2]{%
    % usage: \rightundertwist(bottomCoord)[nameForTopCoord]
    \draw (#1)
        to[out=90, in=180] ++(8pt,10pt)
        to[out=0,in=90] ++(5pt,-5pt)
        coordinate (twist);
    \draw[overdraw=4pt] (twist)
        to[out=270,in=0] ++(-5pt,-5pt)
        to[out=180,in=270] ++(-8pt,10pt)
        coordinate (#2);
}
\newcommand\leftovertwist{}
\def\leftovertwist(#1)[#2]{%
    % usage: \leftovertwist(bottomCoord)[nameForTopCoord]
    \coordinate[above=10pt of #1] (#2);
    \draw (#2)
        to[out=270,in=0] ++(-8pt,-10pt)
        to[out=180,in=270] ++(-5pt,5pt)
        coordinate (twist);
    \draw[overdraw=4pt] (twist)
        to[out=90,in=180] ++(5pt,5pt)
        to[out=0,in=90] ++(8pt,-10pt);
}
\newcommand\leftundertwist{}
\def\leftundertwist(#1)[#2]{%
    % usage: \leftundertwist(bottomCoord)[nameForTopCoord]
    \draw (#1)
        to[out=90,in=0] ++(-8pt,10pt)
        to[out=180,in=90] ++(-5pt,-5pt)
        coordinate (twist);
    \draw[overdraw=4pt] (twist)
        to[out=270,in=180] ++(5pt,-5pt)
        to[out=0,in=270] ++(8pt,10pt)
        coordinate (#2);
}

% =============================================================================
% TYPESETTING TWO-COLUMN NONABELIAN SECTIONS
% =============================================================================
\newcommand\leftcoltitle{\textbf{Concrete case: Group Representation}}
\newcommand\rightcoltitle{\textbf{General case: Tensor Category}}
\newcommand\doublecolcolumnsep{2em}
\newcommand\doublecolvcorrection{-1.5ex}  % skipped at the beginning of each column
\newcommand\doublecolfontsize{\small}

\globalcounter{figure}
\globalcounter{table}
\globalcounter{equation}
\newenvironment{doublecol}{%
    \doublecolfontsize
    \vspace{\baselineskip}
    \setlength{\columnsep}{\doublecolcolumnsep}%
    \backgroundcolor{c[0](.2cm,.2cm)(.2cm,.2cm)}{bg_left}%
    \backgroundcolor{c[1](.2cm,.2cm)(.2cm,.2cm)}{bg_right}%
    \paracol{2}%
    \vspace{\doublecolvcorrection}%
    \leftcoltitle%
    \par%
    \newcommand{\colswitch}{%
        \switchcolumn%
        \vspace{\doublecolvcorrection}%
        \rightcoltitle%
        \par%
    }%
}{%
    \endparacol%
}

% star: without titles
\newenvironment{doublecol*}{%
    \doublecolfontsize
    \vspace{\baselineskip}
    \setlength{\columnsep}{\doublecolcolumnsep}%
    \backgroundcolor{c[0](.2cm,.2cm)(.2cm,.2cm)}{bg_left}%
    \backgroundcolor{c[1](.2cm,.2cm)(.2cm,.2cm)}{bg_right}%
    \paracol{2}%
    \vspace{\doublecolvcorrection}%
    \newcommand{\colswitch}{%
        \switchcolumn%
        \vspace{\doublecolvcorrection}%
    }%
}{%
    \endparacol%
}
\newenvironment{extendrightcol}{%
    \doublecolfontsize
    \setlength{\columnsep}{\doublecolcolumnsep}%
    % separator
    {%
        \vspace{1.5mm}
        \paracol{2}%
        \backgroundcolor{c[0](.2cm,0cm)(.2cm,0cm)}{white}%
        \backgroundcolor{c[1](.2cm,0cm)(.2cm,0cm)}{bg_right}%
        \vspace{3ex}%
        \switchcolumn%
        \vspace{3ex}%
        \endparacol%
    }%
    % full-width
    \paracol{1}%
    \backgroundcolor{c[0](.2cm,.2cm)(.2cm,.2cm)}{bg_right}%
    \vspace{\doublecolvcorrection}
}{%
    \endparacol%
}
\newenvironment{extendleftcol}{%
    \doublecolfontsize
    \setlength{\columnsep}{\doublecolcolumnsep}%
    % separator
    {%
        \vspace{1.5mm}
        \paracol{2}%
        \backgroundcolor{c[0](.2cm,0cm)(.2cm,0cm)}{bg_left}%
        \backgroundcolor{c[1](.2cm,0cm)(.2cm,0cm)}{white}%
        \vspace{3ex}%
        \switchcolumn%
        \vspace{3ex}%
        \endparacol%
    }%
    % full-width
    \paracol{1}%
    \backgroundcolor{c[0](.2cm,.2cm)(.2cm,.2cm)}{bg_left}%
    \vspace{\doublecolvcorrection}
}{%
    \endparacol%
}

% Adjust itemize to give decent spacing in half columns
\newenvironment{halfcolitemize}{%
    \itemize[topsep=.5ex,parsep=1ex,itemsep=.5ex,leftmargin=1.5em]
}{%
    \enditemize
}

% =============================================================================
% CUSTOM ENVIRONMENTS
% =============================================================================
\newenvironment{framedfigure}{\figure[ht]\framed\centering}{\endframed\endfigure}

% =============================================================================
% CUSTOM COMMANDS
% =============================================================================
\newcommand{\FUTURE}[1]{\todo[inline, color=blue!30]{(todo): #1}}
% \newcommand{\FUTURE}[1]{}
\newcommand{\TODO}[1]{\todo[inline]{TODO: #1}}
\newcommand{\FIXME}[1]{\todo[inline, color=red]{FIXME: #1}}

% =============================================================================
% MATH COMMANDS
% =============================================================================
% Misc
\newcommand{\doubletilde}[1]{\tilde{\raisebox{0pt}[0.85\height]{$\tilde{#1}$}}}
\DeclareMathOperator*{\argmin}{arg\,min}
\DeclareMathOperator*{\argmax}{arg\,max}
\DeclareMathOperator*{\tensortr}{TTr}
\newcommand{\nnsum}[2]{\sum_{\langle #1, #2 \rangle}}
\newcommand{\nnprod}[2]{\prod_{\langle #1, #2 \rangle}}

% Autodiff notation
\newcommand{\hadamard}{\circ}
\newcommand{\Off}{\mathcal{O}}  % off diagonal matrix
\newcommand{\dA}{\dd{A}}
\newcommand{\dU}{\dd{U}}
\newcommand{\dS}{\dd{S}}
\newcommand{\dV}{\dd{V}}
\newcommand{\dX}{\dd{X}}
\newcommand{\dY}{\dd{Y}}
\newcommand{\dZ}{\dd{Z}}
\newcommand{\dP}{\dd{P}}
\newcommand{\dC}{\dd{C}}
\newcommand{\dD}{\dd{D}}
\newcommand{\dQ}{\dd{Q}}
\newcommand{\dR}{\dd{R}}
\newcommand{\G}[1]{\Gamma_{#1}}
\newcommand{\Gconj}[1]{\G{\conj{#1}}}
\newcommand{\GcAs}{\Gconj{A}^\mathrm{S}}
\newcommand{\GcAuo}{\Gconj{A}^\mathrm{Uo}}
\newcommand{\GcAvo}{\Gconj{A}^\mathrm{Vo}}
\newcommand{\GcAd}{\Gconj{A}^\mathrm{diag}}
\newcommand{\GcAtr}{\Gconj{A}^\mathrm{tr}}

% Variables i may want to rename a lot
\newcommand{\economicRank}{k}  % e.g. min(m, n) in SVD
\newcommand{\numericalRank}{k}  % e.g. in randNLA context, number of singular values that are non-zero in machine precision
\newcommand{\truncatedRank}{\chi}  % rank of a truncated factorization, e.g truncated SVD

% Special "variables"
\newcommand{\ee}{\ensuremath \mathrm{e}}  % eulers number
\newcommand{\eto}[1]{\ee^{#1}}
\newcommand{\im}{\ensuremath \mathrm{i}}  % imaginary unit i
\newcommand{\dd}{\ensuremath \mathrm{d}}  % differential d
\newcommand{\eye}{\ensuremath \mathbbm{1}}  % identity
\newcommand{\bigO}{\ensuremath \mathrm{O}}  % identity

% Delimiters:
% Helper macros: use \left \right \middle delimiter versions in displaymode, normal otherwise
\newcommand{\twoDelimiters}[3]{\ensuremath%
    \mathchoice{\left#1 #2 \right#3}{#1 #2 #3}{#1 #2 #3}{#1 #2 #3}%
}
\newcommand{\threeDelimiters}[5]{\ensuremath%
    \mathchoice{%
        \left#1 #2 \middle#3 #4 \right#5}{%
        #1 #2 #3 #4 #5}{#1 #2 #3 #4 #5}{#1 #2 #3 #4 #5}
}
\newcommand{\fourDelimiters}[7]{\ensuremath%
    \mathchoice{%
        \left#1 #2 \middle#3 #4 \middle#5 #6 \right#7
    }{%
        #1 #2 #3 #4 #5 #6 #7}{#1 #2 #3 #4 #5 #6 #7}{#1 #2 #3 #4 #5 #6 #7}
}
% Bra-Ket notation
\newcommand{\ket}[1]{\twoDelimiters{\vert}{#1}{\rangle}}
\newcommand{\bra}[1]{\twoDelimiters{\langle}{#1}{\vert}}
\newcommand{\braket}[2]{\threeDelimiters{\langle}{#1}{\vert}{#2}{\rangle}}
\newcommand{\ketbra}[2]{\fourDelimiters{\vert}{#1}{\rangle}{}{\langle}{#2}{\vert}}
\newcommand{\braopket}[3]{\fourDelimiters{\langle}{#1}{\vert}{#2}{\vert}{#3}{\rangle}}
%
\newcommand{\psiket}{\ket{\psi}}
\newcommand{\psibra}{\bra{\psi}}
\newcommand{\GSket}{\ket{\text{GS}}}
\newcommand{\GSbra}{\bra{\text{GS}}}
\newcommand{\phiket}{\ket{\phi}}
\newcommand{\phibra}{\bra{\phi}}
% Brackets and Parentheses
\newcommand{\rBr}[1]{\twoDelimiters{(}{#1}{)}}  % round ()
\newcommand{\sBr}[1]{\twoDelimiters{[}{#1}{]}}  % square []
\newcommand{\cBr}[1]{\twoDelimiters{\lbrace}{#1}{\rbrace}}  % curly {}
% Ceil and Floor
\DeclarePairedDelimiter\ceil{\lceil}{\rceil}
\DeclarePairedDelimiter\floor{\lfloor}{\rfloor}
% norms
\newcommand{\abs}[1]{\twoDelimiters{\vert}{#1}{\vert}}
\newcommand{\norm}[1]{\twoDelimiters{\Vert}{#1}{\Vert}}
% \DeclarePairedDelimiter\abs{\vert}{\vert}
\newcommand{\Fnorm}[1]{\twoDelimiters{\Vert}{#1}{\Vert_\mathrm{F}}}
\newcommand{\Fprod}[2]{\braket{#1}{#2}_\mathrm{F}}

% Special Sets
\newcommand{\set}[1]{\twoDelimiters{\lbrace}{#1}{\rbrace}}
\newcommand{\setdef}[2]{\threeDelimiters{\lbrace}{#1}{\vert}{#2}{\rbrace}}
\newcommand{\Nbb}{\mathbbm{N}}
\newcommand{\Zbb}{\mathbbm{Z}}
\newcommand{\Qbb}{\mathbbm{Q}}
\newcommand{\Rbb}{\mathbbm{R}}
\newcommand{\Cbb}{\mathbb{C}}
\newcommand{\hilbert}{\ensuremath \mathcal{H}}
% named groups
\newcommand{\U}[1]{\ensuremath \mathrm{U}(#1)}
\newcommand{\SU}[1]{\ensuremath \mathrm{SU}(#1)}
% named Lie algebras
\newcommand{\su}[1]{\ensuremath \mathfrak{su}(#1)}

% Binary symbols
\newcommand{\compose}{\ensuremath \circ}  % map composition
\newcommand{\matprod}{\ensuremath \cdot}  % matrix product
\newcommand{\elemprod}{\ensuremath \circ}  % elementwise product
\newcommand{\tensorprod}{\ensuremath \otimes}  % tensor product

% Decorations on symbols
\newcommand{\conj}[1]{\ensuremath \overline{#1}}
\newcommand{\hconj}[1]{\ensuremath {#1}^\dagger}
\newcommand{\transpT}{\text{T}}
\newcommand{\transp}[1]{\ensuremath {#1}^{\transpT}}
\newcommand{\inv}[1]{\ensuremath {#1}^{-1}}
\newcommand{\dualspace}[1]{\ensuremath {#1}^\star}
\newcommand{\doubledualspace}[1]{\ensuremath {#1}^{\star\star}}
\newcommand{\dualsector}[1]{\ensuremath \overline{#1}}
\newcommand{\doubledualsector}[1]{\ensuremath \overline{\overline{#1}}}

% Texts in equation
\newcommand{\const}{\ensuremath \mathrm{const.}}
\newcommand{\pconst}{\ensuremath + \const}
\newcommand{\mconst}{\ensuremath - \const}
\newcommand{\cc}{\ensuremath \mathrm{c.c.}}
\newcommand{\pcc}{\ensuremath + \cc}
\newcommand{\mcc}{\ensuremath - \cc}
\newcommand{\hc}{\ensuremath \mathrm{h.c.}}
\newcommand{\phc}{\ensuremath + \hc}
\newcommand{\mhc}{\ensuremath - \hc}
\newcommand{\RHS}{\mathrm{RHS}}
\newcommand{\LHS}{\mathrm{LHS}}

% Text Operators
\newcommand{\Homset}[2]{\ensuremath{\mathrm{Hom}\left(#1,\; #2\right)}}
\newcommand{\HomsetArg}[3]{\ensuremath{\mathrm{Hom}_{#1}\left(#2,\; #3\right)}}
\newcommand{\Endset}[1]{\ensuremath{\mathrm{End}\left(#1\right)}}
\newcommand{\EndsetArg}[2]{\ensuremath{\mathrm{Hom}_{#1}\left(#2\right)}}
\newcommand{\minArg}[2]{\ensuremath \underset{#1}{\min} \left( #2 \right)}
\newcommand{\maxArg}[2]{\ensuremath \underset{#1}{\max} \left( #2 \right)}
\renewcommand{\Re}[1]{\ensuremath{\mathrm{Re}\!\left[ #1 \right]}}
\renewcommand{\Im}[1]{\ensuremath{\mathrm{Im}\!\left[ #1 \right]}}
\newcommand{\tr}[1]{\ensuremath{\mathrm{Tr}\left( {#1} \right)}}
\newcommand{\trArg}[2]{\ensuremath{\mathrm{Tr}_{#1}\left( {#2} \right)}}

% Functions / Maps
\newcommand{\isoTo}{\ensuremath{\xrightarrow{\cong}}}
\newcommand{\funcDef}[5]{\ensuremath #1 \; : \; #2 \to #3 \; , \; #4 \mapsto #5}
\newcommand{\funcdef}[3]{\ensuremath #1 \; : \; #2 \to #3}
\newcommand{\funcdefIso}[3]{\ensuremath #1 \; : \; #2 \isoTo #3}

% Vectors
\newif\ifvectorsbold
\vectorsboldtrue

\ifvectorsbold
    \renewcommand{\vec}[1]{\boldsymbol{#1}}
    \newcommand{\vect}[1]{\ensuremath\boldsymbol{#1}}  % vector
    \newcommand{\uvect}[1]{\ensuremath\hat{\boldsymbol{#1}}}  % unit vector
    \newcommand{\kanon}[1]{\ensuremath\hat{\boldsymbol{e}}_{#1}}  % canonical basis vector
\else
    \newcommand{\vect}[1]{\ensuremath\vec}  % vector
    \newcommand{\uvect}[1]{\ensuremath\hat{#1}}  % unit vector
    \newcommand{\kanon}[1]{\ensuremath\hat{e}_{#1}}  % canonical basis vector
\fi

% Equality signs and delimiters
% \newcommand{\refeq}[1]{\overset{\eqref{#1}}{=}}


% Derivatives (operator as fraction d/dx)
\newcommand{\diff}[1]{\ensuremath \frac{\dd}{\dd {#1}}}
\newcommand{\ndiff}[2]{\ensuremath \frac{\dd^{#2}}{\dd {#1}^{#2}}}
\newcommand{\pdiff}[1]{\ensuremath \frac{\partial}{\partial {#1}}}
\newcommand{\npdiff}[2]{\ensuremath \frac{\partial^{#2}}{\partial {#1}^{#2}}}

% Derivaties (whole derivative as fraction df/dx
\newcommand{\Diff}[2]{\ensuremath \frac{\dd {#1}}{\dd {#2}}}
\newcommand{\nDiff}[3]{\ensuremath \frac{\dd^{#3} {#1}}{\dd {#2}^{#3}}}
\newcommand{\pDiff}[2]{\ensuremath \frac{\partial {#1}}{\partial {#2}}}
\newcommand{\npDiff}[3]{\ensuremath \frac{\partial^{#3} {#1}}{\partial {#2}^{#3}}}

% Integration
\renewcommand{\d}[1]{\ensuremath \,\! \dd {#1} \,\! }  % differential d{x} -> dx

% Category theory
\newcommand{\blank}{{-}}
\newcommand{\catC}{\mathbf{C}}
\newcommand{\catD}{\mathbf{D}}
\newcommand{\catRepG}{\mathbf{Rep}_G}
\newcommand{\catFerm}{\mathbf{Ferm}}
\newcommand{\catFib}{\mathbf{Fib}}
\newcommand{\objset}[1]{\mathrm{Ob}(#1)}
\newcommand{\morphset}[3]{#1(#2, #3)}
\newcommand{\morphism}[3]{#1: #2 \to #3}
\newcommand{\isomorphism}[3]{#1: #2 \isoTo #3}
\newcommand{\naturaltrafo}[3]{#1: #2 \Longrightarrow #3}
\newcommand{\id}[1]{\mathrm{id}_{#1}}
\newcommand{\cupmap}[1]{\eta_{#1}}
\newcommand{\capmap}[1]{\varepsilon_{#1}}
