In this chapter we provide extra material for the discussions regarding symmetries, in section~\ref{sec:tensornets:symmetries} and chapter~\ref{ch:nonabelian}.
%
In section~\ref{sec:topo_data:review_rep_thry}, we review the basics of group representations, as well as prominent results such as Schur's lemma.
%
In the subsequent sections, we discuss common symmetry groups, and give the topological data needed to enforce these symmetries in the framework of chapter~\ref{ch:nonabelian}, using fusion trees and their manipulations.
%
We cover the abelian groups $\Zbb_N$ and $\U{1}$ in section~\ref{sec:topo_data:ZN} and \ref{sec:topo_data:U1}, the nonabelian group $\SU{2}$ in section~\ref{sec:topo_data:SU2}, fermionic grading in~\ref{sec:topo_data:ferm} and Fibonnaci anyons in section~\ref{sec:topo_data:fib}.
%
We discuss combining symmetries in section~\ref{sec:topo_data:product}.


\section{Review: Representation theory}
\label{sec:topo_data:review_rep_thry}

Let us review some basics from the representation theory of groups.

A (faithful) \emph{representation} $U$ of a group $G$ on a vector space $V$ is a group homomorphism $U: G \to \Homset{V}{V}$.
%
This means a group representation assigns to every group element $g\in G$ a linear map $U(g): V \to V$, such that the group structure is preserved, i.e.~for $g, h \in G$ we have $U(g h) = U(g) U(h)$.
%
We suggestively use the symbol $U$ for the representation, since we want to think of it as unitary.

A linear map $f: V \to W$ is \emph{equivariant} between representations $U$ on $V$ and $U'$ on $W$ if $f \compose U(g) = U'(g) \compose f$ for all $g\in G$.
%
Two representations $U$ and $\tilde{U}$ of the same group $G$ on vector spaces $V$ and $\tilde{V}$ respectively, are \emph{equivalent} if there is an equivariant vector space isomorphism $S: V \to \tilde{V}$, i.e.~an invertible linear map such that $\tilde{U}(g) = S \compose U(g) \compose S^{-1}$ for all $g\in G$.
%
In that case, we write $U \cong \tilde{U}$.
%
We can think of equivalence as the criterion for a basis change between vector spaces $V \to \tilde{V}$ with associated representations, such that equivariance is preserved, meaning if $f: V \to W$ is equivariant, so is $f \compose S^{-1}$.

A representation $U$ on a vectorspace $V$ is \emph{reducible}, if there is a non-trivial proper subspace $0 \neq W \subset V$ which is invariant under $U$, that is if $U(g)(w) \in W$ for all $w\in W$ and $g \in G$.
%
The trivial subspace $W = 0$ and the full space itself $W = V$ are always invariant and are therefore excluded.
%
A representation is \emph{irreducible} if there is no such $W$.
%
We refer to irreducible representations as \emph{irreps} for short.
%
While reducibility is suggestively named, we are not in general guaranteed that they can actually be reduced.
%
Let us first characterize what it means to be ``reduced", namely as a \emph{direct sum}.

The direct sum $V \oplus W$ of vector spaces $V, W$ is given by $\setdef{(v, w)}{v\in V, w \in W}$ with componentwise addition and scalar multiplication.
%
If $V, W$ are inner product spaces, we can equip $V \oplus W$ with the inner product $\braket{(v_1, w_1)}{(v_2, w_2)} = \braket{v_1}{v_2}_V + \braket{w_1}{w_2}_W$.
%
The direct sum $f \oplus g$ of linear maps $f_1 : V_1 \to V_2$ and $g : W_1 \to W_2$ is given by componentwise application $f \oplus g : V_1 \oplus W_1 \to W_1 \oplus W_2 , (v, w) \mapsto (f(v), g(w))$ and in an appropriate basis, its matrix representation is the block-diagonal matrix formed from the matrix representations of $f, g$.
%
Finally, the direct sum $U \oplus \tilde{U}$ of representations $U$ and $\tilde U$ of a group $G$ on vector spaces $V$ and $\tilde V$ respectively is a representation on $V \oplus \tilde{V}$ given by $(U \oplus \tilde{U})(g) = U(g) \oplus \tilde{U}(g)$, i.e.~by the componentwise direct sum of linear maps.
%
Generalizations of these definitions to direct sums of multiple spaces / maps / representations are straight-forward.
    
\textbf{Irrep decomposition:}~~
We can now reduce general representations, if we include the additional assumption that they are unitary;
%
Any unitary representation $U$ of a group $G$ on a finite-dimensional vector space $V$ is equivalent to a finite direct sum $U \cong \bigoplus_{n=1}^N U_n$ of irreps $U_n$ of $G$.

\begin{proof}
    If $U$ is irreducible, the claim is trivially fulfilled.
    %
    Let us now assume $U$ is reducible and let $0 \neq W \subset V$ be an invariant subspace.
    
    Consider the orthogonal complement $W^\perp = \setdef{v \in V}{\braket{v}{w} = 0 ~\forall w \in W}$.
    %
    For any $g\in G$, $U(g)$ is unitary and in particular injective.
    %
    Thus the restriction $U_1(g): W \to W, w \mapsto U(g)(w)$ is (a) well-defined since $W$ is invariant and (b) also injective.
    %
    Since it is a linear map between vector spaces of equal finite dimension, it is also surjective.
    %
    Now let $x \in W^\perp$ and $w \in W$.
    %
    Because of surjectivity, there is a $w' \in W$ s.t. $U(g)(w') = w$, which means that $U(g)(x)$ is in $W^\perp$ since $\braket{U(g)(x)}{w} = \braket{U(g)(x)}{U(g)(w')} = \braket{x}{w'} = 0$.
    %
    This shows invariance of $W^\perp$ and allows us to define the restriction $U_2(g): W^\perp \to W^\perp , x \mapsto U(g)(x)$.
    %
    Both restrictions $U_{1/2}$ are unitary by construction.

    Now consider the vector space isomorphism $S: V \to W \oplus W^\perp, v \mapsto (P_W(v), v - P_W(v))$, where $P_W$ is the projector on the subspace $W$.
    %
    The inverse is simply $S^{-1} : W \oplus W^\perp \to V , (w, w') \mapsto w + w'$.
    %
    We can conclude equivariance for all $g\in G$
    \begin{align}
        \begin{split}
            (S \compose U(g) \compose S^{-1})(w, w')
            &= S \compose (U(g)(w) + U(g)(w'))
            = S \compose (U_1(g)(w) + U_2(g)(w'))
            \\
            &= (U_1(g)(w), U_2(g)(w')
            = (U_1 \oplus U_2)(g)(w, w')
            ~,
        \end{split}
    \end{align}
    where we used that $U(g)(w) = U_1(g)(w)$ for $w \in W$ and similarly for $U(g)(w) = U_2(g)(w')$ for $w \in W^\perp$. This establishes $U \cong U_1 \oplus U_2$.

    This procedure can now be iterated for $U_1$ and/or $U_2$, until all components are irreducible.
    %
    This terminates after a finite number of steps, as the dimension of the representation spaces starts at the finite $\dim V$ and strictly decreases at each step.
\end{proof}

The irrep decomposition is in general not unique, as any irrep $U_n$ may be replaced by an equivalent irrep $\tilde{U}_n \cong U_n$.
%
It is therefore convenient to choose one representative for each equivalence class of irreps.
%
We write $U_\mathbf{a}$ for such a representative and call it a \emph{sector}.
%
For any group, the \emph{trivial representation} $U_\mathbf{0} : g \mapsto 1$ on the one-dimensional space $\Cbb$ is irreducible, and can be chosen as the representative of the \emph{trivial sector}, also called the symmetric sector.
%
Let us write $\mathcal{S}_G$ for the set of sector labels such that every irrep $U$ of a group $G$ has exactly one $\mathbf{a} \in \mathcal{S}_G$ such that $U \cong U_\mathbf{a}$.
%
Let us consider a few examples, which we derive and discuss in more detail in the following sections.
%
We find integer labels $\mathcal{S}_{\U{1}} = \Zbb$ for $\U{1}$, and $\mathcal{S}_{\Zbb_N} = \Zbb_N$ for $\Zbb_N$.
%
For multiple symmetries, i.e.~for the product group $G \times H$ we find $\mathcal{S}_{G \times H} = \mathcal{S}_{G} \times \mathcal{S}_{H} = \setdef{(\mathbf{a}, \mathbf{b})}{\mathbf{a} \in \mathcal{S}_{G}, \mathbf{b}\in \mathcal{S}_{H}}$, i.e.~tuples of the respective individual sector labels.

Now, unitarity is not a strong requirement.
%
If $G$ is a finite group, a compact Lie group, or any other group that admits a right-invariant Haar measure, any of its representations $D$ on a space $V$ is equivalent to a unitary representation $U$ on the same representation space $V$.

\begin{proof}
    Let $D$ be a representation of $G$ on $V$ and $\braket{\blank}{\blank}$ the inner product on $V$.
    %
    First, we construct a different inner product $\braket{\blank}{\blank}_D$ on $V$ w.r.t.~which $D$ is unitary.
    %
    Under the assumptions above, we have some integral measure $\int_G \d{x}$ on the group $G$, which is e.g.~given by $\int_G \d{x} f(x) =\tfrac{1}{\abs{G}} \sum_{x\in G} f(x)$ for finite groups $G$ and group functions $f: G \to \Cbb$.
    %
    It is a (right-) invariant measure, meaning $\int_G \d{x} f(xy) = \int_G \d{x} f(x)$ for any $y \in G$.
    %
    Using this measure we define a different inner product for $v, w \in V$ as
    \begin{equation}
        \braket{v}{w}_D := \int \d{x} \braket{D(x)(v)}{D(x)(w)}
        .
    \end{equation}
    It is easy to check that it is indeed an inner product (i.e.~hermitian, linear and positive).
    %
    Now for any $g \in G$ and $v, w \in V$ we have
    \begin{equation}
        \braket{D(g)(v)}{D(g)(w)}_D = \int \d{x} \braket{D(xg)(v)}{D(xg)(w)} = \braket{v}{w}_D
    \end{equation}
    because of the invariance of the measure, meaning $D$ is indeed unitary w.r.t.~$\braket{\blank}{\blank}_D$.
    
    Now, there is an isomorphism $S: V \to V$ of inner product spaces $(V, \braket{\blank}{\blank}_D)$ and $(V, \braket{\blank}{\blank})$, meaning $\braket{S(v)}{S(w)} = \braket{v}{w}_D$.
    %
    It can e.g.~be constructed by mapping an orthonormal basis in $(V, \braket{\blank}{\blank})$ to an orthonormal basis in $(V, \braket{\blank}{\blank}_D)$ and linearly extending.
    %
    Define the representation $U$ of $G$ on $V$ via $U(g) := S \compose D(g) \compose S^{-1}$, which is equivalent to $D$ by construction.
    %
    Now for $g\in G$ and $v, w \in V$ we have
    \begin{align}
    \begin{split}
        \braket{U(g)(v)}{U(g)(w)}
        &= \braket{(D(g) \compose S^{-1})(v)}{(D(g) \compose S^{-1})(w)}_D
        = \braket{S^{-1}(v)}{S^{-1}(w)}_D
        \\
        &= \braket{v}{w},
    \end{split}
    \end{align}
    where we used the defining property of $S$, that $D$ is unitary w.r.t.~$\braket{\blank}{\blank}_D$ and again the defining property of $S$.
    %
    Thus we find that $U$ is unitary w.r.t.~the inner product $\braket{\blank}{\blank}$ on $V$ and equivalent to $D$.
\end{proof}

Thus for finite groups or compact Lie groups, which covers virtually all cases for symmetry groups in condensed matter physics, we may simply assume a representation is unitary, up to equivalence.
%
As a corollary, \emph{any} representation -- unitary or not -- of a finite group or a compact Lie group is equivalent to a direct sum of irreps.

The main result from representation theory that allows an efficient representation of symmetric tensors is \emph{Schur's Lemma}, which comes in two parts.

\textbf{Schur's lemma part 1:}~~
Let $U$ and $\tilde{U}$ \underline{in}equivalent irreps of a group $G$ on finite dimensional vector spaces $V$ and $\tilde{V}$ respectively.
%
Now if $M : V \to \tilde{V}$ is equivariant, meaning linear and $\tilde{U}(g) \compose M = M \compose U(g)$ for all $g\in G$, then $M=0$.

\begin{proof}
    Distinguish three cases regarding dimensionality.

    In case 1, assume $\dim V < \dim \tilde{V}$. Consider the subspace $M(V) := \setdef{M(v)}{v\in V} \subseteq \tilde{V}$ and let $x = M(v) \in M(V)$.
    %
    Then $\tilde{U}(g)(x) = (M \compose U(g))(v) \in M(V)$ for all $g\in G$.
    %
    Thus $M(V)$ is an invariant subspace of $\tilde{U}$.
    %
    Since $\tilde{U}$ is an irrep, we have either $M(V) = \tilde{V}$ or $M(V) = 0$.
    %
    The former is impossible for dimensional reasons and thus $M(v) = 0$ for all $v\in V$, meaning $M=0$.

    In case 2, assume $\dim V > \dim \tilde{V}$.
    %
    Here we can reason similarly that $W := \mathrm{kernel}(M) = \setdef{v \in V}{M(v) = 0}$ is an invariant subspace of $U$, and because it is an irrep conclude that $W=V$ and thus $M=0$.

    In case 3, if $\dim V = \dim \tilde{V}$, if either $M(V) = \tilde{V}$ or $W = V$, we find $M=0$ by the same logic as in the previous cases.
    %
    Thus, the remaining case for which we have not yet concluded $M=0$ is $M(V) = \tilde{V}$ and $\mathrm{kernel}(M) = 0$. This would however imply that $M$ is both surjective and injective and thus a vector space isomorphism. Since it is also equivariant, it would witness an equivalence $U \cong \tilde{U}$, which contradicts the assumption.
\end{proof}

In other words, there are no intertwiners (linear $M$ with the above property) between inequivalent irreps.
%
Even between equivalent irreps, the intertwiners are constrained as follows.

\textbf{Schur's lemma part 2:}~~
Let $U$ an irrep of $G$ on a vector space $V$ over an algebraically closed field (e.g.~over $\Cbb$), and $M : V \to V$ equivariant.
%
Then $M = \lambda ~ \id{V}$ for some scalar $\lambda \in\Cbb$.

\begin{proof}
    Since the field is algebraically closed, $M$ has at least one eigenvalue $\lambda$.
    Let $0 \neq v \in V$ be a corresponding eigenvector. Now consider the eigenspace $E_\lambda = \setdef{v\in V}{M(v) = \lambda v}$. 
    It is an invariant subspace since for any $v\in E_\lambda$ and $g\in G$ we have
    $M( U(g)(v) ) = U(g)( M(v) ) = U(g)(\lambda v) = \lambda U(g)(v)$ and thus $U(g)(v) \in E_\lambda$.
    Since $U$ is an irrep, we have either $E_\lambda = 0$ or $E_\lambda = V$.
    Since there is an eigenvector $v \neq 0$, the former is ruled out and we find $M(v) = \lambda v$ for all $v\in V$, or equivalently $M = \lambda \id{V}$.
\end{proof}

We find as a corollary that irreps of an abelian group $G$ are one dimensional, if the underlying field is algebraically closed.

\begin{proof}
    Let $U$ be an irrep on a vector space $V$.
    Since $G$ is abelian, we have $U(g) U(h) = U(gh) = U(hg) = U(h) U(g)$ for all $g, h \in G$.
    Thus Schurs Lemma part 2 applies with $M = U(h)$ and we find $U(g) = \lambda \id{V}$ for some scalar $\lambda$, which may depend on $g$.
    This means that any subspace of $V$ is invariant under $U$, which contradicts irreducibility of $U$ unless there are no non-trivial proper subspaces, i.e.~unless $\dim V = 1$.
\end{proof}

To exploit Schur's Lemma, it is convenient to use a basis $\set{e_i}$ for every vector space such that the group representation is not only equivalent but actually equal to a direct sum of irreps.
%
This is done by employing the equivariant isomorphism -- a basis change -- guaranteed by the equivalence to a direct sum of irreps.
%
For abelian groups, where all irreps are one-dimensional, this effectively assigns an irrep label $\mathbf{a}_i$ to every basis element $e_i$.

This imposes a sparsity structure on the matrix representations of equivariant maps, as follows.
%
Let $f: V \to \tilde{V}$ be equivariant between representations $U = \bigoplus_{i=1}^{M} U_i$ on $V$ and $\tilde{U} = \bigoplus_{j=1}^{N} \tilde{U}_j$ on $\tilde{V}$ of an \emph{abelian} symmetry group $G$, where $U_i, \tilde{U}_j$ are irreps and $M = \dim V$, as well as $N = \dim \tilde{V}$.
%
For the matrix elements $f_{ij} = \braket{\tilde{e}_i}{f(e_j)}$ in the computational bases $\set{e_j}$ of $V$ and $\set{\tilde{e}_i}$ of $\tilde{V}$, equivariance $f = \tilde{U}^\dagger(g) \compose f \compose U(g)$ translates to
\begin{equation}
    f_{ij}
    = \braket{\tilde{U}(g)(\tilde{e}_i)}{f(U(g)(e_j))}
    = \braket{\tilde{U}_i(g) \tilde{e}_i}{f(U_j(g) e_j )}
    = \tilde{U}^\dagger_i(g) f_{ij} U_j(g)
    ,
\end{equation}
where we used that $\tilde{U}_i(g)$ and $U_j(g)$ are just numbers and that $f$ is linear.
%
Thus for fixed indices $i = 1,\dots N$ and $j = 1,\dots, M$ the entry $f_{ij}$ of the matrix representation is an equivariant map $\Cbb \to \Cbb$ between $U_i$ and $\tilde{U}_j$ and by Schurs lemma part 1, we have $f_{ij} = 0$ if $U_i \ncong \tilde{U}_j$ are inequivalent.
%
Thus, if we sort the basis elements by sector, that is by equivalence class of the irreps, we find a block diagonal structure for the matrix representation of $f$, and the allowed blocks are between basis elements $e_i \in V$ and $e_j \in W$ such that the irreps $U_i \cong \tilde{U}_j$ are in the same sector.

To apply the same idea to tensors, we need to define the product representation on the tensor product of vector spaces.
%
The tensor product $U_1 \otimes U_2$ of representations $U_i$ on vector spaces $V_i$ is a representation on the tensor product $V_1 \otimes V_2$ which is given by $(U_1 \otimes U_2)(g) = U_1(g) \otimes U_2(g)$.
%
Note that for abelian groups, the product of irreps, which are one-dimensional, is also one-dimensional and thus equivalent to a single irrep.
%
This defines fusion rules of sectors, which we write as $\mathbf{a} + \mathbf{b} = \mathbf{c}$ if $U_\mathbf{a} \otimes U_\mathbf{b} \cong U_\mathbf{c}$.
%
For the group $G = \U{1}$, for example, the sector labels are integers $\mathbf{a} \in \Zbb$ and the fusion rules are regular addition of labels.
%
For a $G = \Zbb_N$ group, the sector labels are from $\mathbf{a} \in \Zbb_N$ and the fusion rules are addition modulo $N$, as the notation suggests.
%
For products of groups, the fusion rules are componentwise.


The final ingredient is the natural representation on the dual space $\dualspace{V}$ -- the space of bra vectors -- given a representation on a ``ket space" $V$.
%
Given a representation $U$ of a group $G$ on a space $V$ with components $\ket{\psi} \mapsto U(g) \ket{\psi}$, the contragradient representation $\bar{U}$ is a representation on the dual space $\dualspace{V}$ given by $\bar{U}(g) = U(g^{-1})^\transpT$, which is the map $\bra{\phi} \mapsto \bra{\phi} U^\dagger(g)$.
%
As a result, acting on both a bra and a ket vector with the respective representation leaves the inner products $\rBr{\bra{\phi} U^\dagger(g) }\rBr{U(g) \ket{\psi}} = \braket{\phi}{\psi}$ invariant.

For abelian groups, the contragradient representation of an irrep, and in particular of a sector $\mathbf{a}$ is again one-dimensional, thus irreducible and equivalent to a single sector.
%
This gives rise to the notion of the ``opposite" sector and we write $-\mathbf{a}$ for that sector label such that $\bar{U}_\mathbf{a} \cong U_{-\mathbf{a}}$.
%
We have an equivalence $\bar{U}_\mathbf{a} \otimes U_\mathbf{a} \cong U_\mathbf{0}$ to the trivial sector, where the isomorphism is given by linear extension of $S: \bra{\phi} \otimes \ket{\psi} \mapsto \braket{\phi}{\psi}$.
%
In particular, this establishes $\mathbf{a} + (-\mathbf{a}) = \mathbf{0}$, justifying the notation for the opposite sector.
%
For our examples $\U{1}$ ($\Zbb_N$), $-\mathbf{a}$ is actually the negative integer of $\mathbf{a}$ (modulo $N$), and for product groups it is componentwise.


% =======================================================================================
% =======================================================================================
% =======================================================================================
\section[Cyclic Groups ZN]{Cyclic Groups $\mathbb{Z}_N$}
\label{sec:topo_data:ZN}

In the following section, here for $G = \Zbb_N$, we give relevant properties, and in particular the topological data for a number of symmetries relevant in condensed matter physics.

The group $\Zbb_N$ for an integer $N > 1$ is given by the numbers $\Zbb_N = \set{0, 1, \dots, N - 1}$ with addition modulo $N$.
%
Since it is an abelian group, all of its irreps are one-dimensional and there are $N$ equivalence classes of irreps.
%
For each $a = 0, \dots, N - 1$, we choose the following representative irrep for the sector $a$;
$U_a(g) : \Cbb \to \Cbb , z \mapsto \eto{2\pi\im \frac{a}{N} g} z$, where $g \in \Zbb_N$.
%
Thus, sectors are labeled by non-negative integers $a \in \mathcal{S} = \set{0,1,\dots, N-1}$. The dual sector is $\dualsector{a} = (-a \mod N)$.
%
The fusion rules are 
\begin{equation}
    a \otimes b \cong (a + b \mod N)
    ,
\end{equation}
where in the following we will drop explicitly writing $\mod N$ and all arithmetic is implicitly $\mod N$.

The N symbol is given by $N^{ab}_c = \delta_{a + b, c}$.
%
For abelian groups, a fusion tree is fully determined by the uncoupled sectors and in the following we only give the values of the symbols for valid fusion channels.

The F symbol is trivial
\begin{equation}
    \big[ F^{abc}_{a + b + c} \big]^{b + c,1,1}_{a + b,1,1} = 1
\end{equation}
where we directly plugged in the only valid sectors, e.g.~$d=a+b+c$ and multiplicity labels $\mu = \nu = \kappa = \lambda = 1$.
%
The R symbol is similarly trivial
\begin{equation}
    \big[ R^{ab}_{a+b} \big]^1_1 = 1
    .
\end{equation}
The set of valid fusion outcomes consists of only a single sector $\mathcal{F}_{a,b} = \set{(a + b \mod N)}$.
%
The C symbol, like the F symbol, is one for the only valid fusion channel
\begin{equation}
    \big[ C^{abc}_{a+b+c} \big]^{a+b,1,1}_{a+c,1,1} = 1
    ~.
\end{equation}
%
The B symbol is similarly
\begin{equation}
    \big[ B^{ab}_{a+b} \big]^1_1 = 1
    ~.
\end{equation}
%
As for any abelian symmetry, all sectors are one-dimensional $d_a = 1$.
%
As for any group symmetry, the twists $\Theta_a = 1$ are trivial.
%
The Frobenius Schur indicators $\chi_a = +1$ are all positive.

The fusion tensors are trivial, given by linear extension of
\begin{equation}
    X^{ab}_{c,1} : 1 \otimes 1 \mapsto 1
\end{equation}
and the Z isomorphisms are also just one, meaning $Z_a: \dualspace{\Cbb} \ni z \mapsto z \in \Cbb$.

% =======================================================================================
% =======================================================================================
% =======================================================================================
\section{The abelian group U(1)}
\label{sec:topo_data:U1}

The abelian group $\U{1}$ is the unit circle $\setdef{z \in \Cbb}{\abs{z} = 1}$ in the complex plane with multiplication as group operation.
%
It has a countably infinite number of sectors, indexed by $a \in \Zbb$, where the representative irrep is given by
\begin{equation}
    U_a(\eto{\im\phi}) : \Cbb \to \Cbb , z \mapsto \eto{\im a \phi} z,
\end{equation}
where $\eto{\im\phi}\in\U{1}$ is a general group element.
%
All results for $\Zbb_N$ groups listed above also hold for $\U{1}$, if we replace the addition modulo $N$ with regular addition.

% =======================================================================================
% =======================================================================================
% =======================================================================================
\section{The non-abelian group SU(2)}
\label{sec:topo_data:SU2}

The compact Lie group $\SU{2}$ is given by complex $2 \times 2$ matrices which are unitary and are special (have determinant one).
%
It has a countably infinite number of sectors, labeled by a half-integer ``total spin" $j \in \tfrac{1}{2}\Zbb$.

For derivations and for concrete expressions of the representation, it is useful to go to the associated Lie algebra $\su{2}$, and later to its complexification.
%
A general element $g \in\SU{2}$ can always be written as $\eto{\ell}$ for some $\ell\in\su{2}$.
%
For the spin $j$ irrep $U_j: \SU{2} \mapsto \Homset{R_j}{R_j}$ of $\SU{2}$, there is a compatible representation $\pi_j : \su{2} \to \Homset{R_j}{R_j}$ on the same representation space $R_j = \Cbb^{2j + 1}$, such that $U_j(g) = \exp (\pi_j(\ell))$.
%
This allows us to build the group irreps from the Lie algebra representations, which are easier to deal with due to the vector space structure.
%
The standard $\Cbb$-basis for the complexified Lie algebra $\mathfrak{su}(2)_\Cbb$ consists of three elements $j_3, j_+, j_-$. Their representations $J_k = \pi_j(j_k)$ are the following operators, given as $(2j + 1) \times (2j + 1)$ matrices

\begin{gather}
    \begin{gathered}
    J_3 = \begin{pmatrix}
        j & & & & \\
        & j - 1 & & & \\
        & & \ddots & & \\
        & & & 1 - j & \\
        & & & & -j
    \end{pmatrix}
    \\
    J_+ = \begin{pmatrix}
        0 & & & & \\
        1 & 0 & & & \\
        & 1 & \ddots & & \\
        & & \ddots & 0 & \\
        & & & 1 & 0
    \end{pmatrix}
    \quad
    J_+ = \begin{pmatrix}
        0 & 1 & & & \\
        & 0 & 1 & & \\
        & & \ddots & \ddots & \\
        & & & 0 & 1 \\
        & & & & 0
    \end{pmatrix}
    ~.
    \end{gathered}
\end{gather}
%
They are related to the standard $\Rbb$-basis of the real Lie algebra $\mathfrak{su}(2)$ as $J_3 = \im L_3$ and $J_\pm = \im L_1 \mp L_2$.
%
Or conversely $L_1 = -\tfrac{\im}{2} (J_+ + J_-)$, $L_2 = -\tfrac{1}{2} (J_+ - J_-)$ and $L_3 = -\im J_3$.
%
Note that the ``real" qualifier of the real Lie algebra refers to the fact that $\mathfrak{su}(2)$ is a real vectorspace, and that linear combinations of its basis elements need to have real coefficients. The representation matrices $L_i$, however, have complex entries and the representation space $R_j$ is a complex vector space.

The $j=0$ case gives us the trivial representation $U_{j=0}(g) : z \mapsto z$.
%
The $j=1/2$ case gives us the faithful representation $U(g) = g$.

The fusion rules are $j_1 \otimes j_2 = \bigoplus_{J=\abs{j_1 - j_2}}^{j_1 + j_2} J$, such that 
\begin{equation}
    N^{ab}_c = \begin{cases}
        1 & c \in\set{\abs{a - b}, \dots, a + b} \\
        0 & \text{else}
    \end{cases}
    ~.
\end{equation}
%
We can directly read off the set of fusion outcomes $\mathcal{F}_{a,b} = \set{\abs{a - b}, \dots, a + b}$.
%
Note that if $a, b$ are either both integer or both fractional, the fusion outcomes are all integer.
%
Conversely if either $a$ or $b$, but not both, are fractional, all fusion outcomes are fractional.
%
Since the N symbol can not take values greater than one, the only possible multiplicity label on a valid fusion tensor is $\mu=1$.

For the fusion tensors, we choose the usual Clebsch-Gordan coefficients, commonly denoted as $\braket{j_1 m_1 j_2 m_2}{J M}$.
%
In the standard z-basis $\setdef{\ket{m}}{m = -j, \dots, j}$ for the representation space $R_j$ of the spin-$j$ irrep, that means
\begin{equation}
    X^{j_1,j_2}_{J,1} = \ket{m_1} \otimes \ket{m_2} \mapsto \sum_{M=-J}^J \braket{J M}{j_1 m_1 j_2 m_2} \ket{M}
    ~.
\end{equation}

The F symbol, as we define it in~\eqref{eq:nonabelian:basics:def_F_symbol}, is related to the Wigner 6j symbol or the Racah W symbol $W(j_1 j_2 J j_3; J_{12} J_{23})$ as follows
\begin{align}
    \begin{split}  
        \big[ F^{abc}_d \big]^{e,1,1}_{f,1,1} 
        &= \sqrt{2 e + 1} \sqrt{2 f + 1} W(a b d c; f e)
        \\
        &= \sqrt{2 e + 1} \sqrt{2 f + 1} (-1)^{a+b+c+d}
            \begin{Bmatrix} a & b & f \\ c & d & e \end{Bmatrix}
        .
    \end{split}
\end{align}
%
For computation and storage schemes for the 6j symbols which give the F symbols and the Wigner 3j symbols, which give the fusion tensors refer e.g.~to reference~\cite{rasch2004}.

%
The R symbol is given by
\begin{equation}
    \big[ R^{ab}_c \big]^1_1 = (-1)^{a + b - c}
    ~.
\end{equation}
Note that the exponent is integer for valid fusion channels $N^{ab}_c > 0$.

The quantum dimensions are $d_a = 2a + 1$.
The Frobenius Schur indicator is given by$\chi_a = (-1)^{(2a)}$.
It is positive for integer spins and negative for half integers, while the twist, like for all group symmetries, is $\Theta_a = +1$.

The Z isomorphism takes the following explicit form in the standard z-basis $\set{\ket{m}}$.
\begin{equation}
    Z_j : \dualspace{R_j} \to R_j , \bra{m} \mapsto \sum_n A^j_{m, n} \ket{n}
    ,
\end{equation}
where
\begin{equation}
    A^j = \begin{pmatrix}
        & & & & & -\chi_j \\
        & & & & \chi_j & \\
        & & & \iddots & & \\
        & & -1 & & & \\
        & 1 & & & & \\
        -1 & & & & &
    \end{pmatrix}
\end{equation}
is a $(2j + 1) \times (2j + 1)$ anti-diagonal matrix with alternating signs, such that its top right entry is $-\chi_j = (-1)^{2j+1}$.
%
We have chosen the phase of the Z ismorphism, such that~\eqref{eq:nonabelian:cup_from_splitting_tensor} holds with the usual phase choice of the Clebsch Gordan coefficients.

\subsection{Derivation of the Z isomorphism}
\label{subsec:topo_data:su2:z_iso_derivation}

For completeness, and to facilitate similar treatment of other Lie groups, we give the full derivation of the above result for the Z isomorphism in the following.
%
First, as a small warning, note that $Z_j$ is a \emph{linear} map from bra vectors to ket vectors.
%
This is rather unusual in physics, where we may have a strong expectation that such maps, such as e.g.~the dagger, are anti-linear.

We show that the map defined above has the necessary properties to be a Z isomorphism.
%
First, it is by definition linear and inherits unitarity from the matrix $A$.
%
It remains to check that $Z_j$ is equivariant.

It is straight-forward to check that $(A^j)^\dagger J_i A^j = -J_i$ for the basis $J_3, J_\pm$ of the complexified Lie algebra.
%
This implies $(A^j)^\dagger L_i A^j = \bar{L}_i$ for the basis of the real Lie algebra, where $\bar{L}$ denotes the elementwise complex conjugate matrix of $L$.

The representation $\sum_i \alpha_i L_i = L = \pi_j(l)$ of a general element $\mathfrak{su}(2) \owns l = \sum_i \alpha_i l_i$ is a real ($\alpha_i \in \Rbb$) linear combination of the $L_i$ and thus also fulfills $(A^j)^\dagger L A^j = \bar{L}$.
%
Therefore the respective representation $U(g)$ of a general element $SU(2) \owns g = \ee^l$ fulfills
\begin{equation}
    (A^j)^\dagger U(g) A^j
    = (A^j)^\dagger \ee^L A^j
    = \ee^{(A^j)^\dagger L A^j}
    = \ee^{\bar{L}}
    = \bar{U}(g)
    ~.
\end{equation}

This is sufficient to show that $Z_j$ is symmetry-preserving, meaning $Z_j \compose U_{\dualspace{(R_j)}}(g) = U_{R_j}(g) \compose Z_j$ for all $g \in SU(2)$.
It is enough to compare the action on a generic basis element;
%
\begin{align}
    \begin{split}
        \text{LHS: }
        \bra{m}
        &\mapsto \bra{m} U^\dagger(g)
        = \sum_{m'} \overline{U}_{m'm} \bra{m'}
        \mapsto \sum_{m'n} \overline{U}_{m'm} A^j_{nm'} \ket{n}
        \\
        &= \sum_{n} \left[A^j \cdot \overline{U}(g)\right]_{nm} \ket{n}
    \end{split}
    \\
    \begin{split}
        \text{RHS: }
        \bra{m}
        &\mapsto \sum_{m'} A^j_{m'm} \ket{m'}
        \mapsto \sum_{m'} A^j_{m'm} U(g) \ket{m'}
        = \sum_{m'n} A^j_{m'm} U_{nm'}(g) \ket{n}
        \\
        &= \sum_{n} \left[U(g) A^j\right]_{nm} \ket{n}
        = \sum_{n} \left[A^j (A^j)^\dagger U(g) A^j\right]_{nm} \ket{n}
        \\
        &= \sum_{n} \left[A^j \overline{U}(g)\right]_{nm} \ket{n}
        % = \text{LHS}(\bra{m})
        ~,
    \end{split}
\end{align}
where we write $U(g) := U_{R_j}(g)$ for the representation on the ket space $R_j$ for readability and $U_{nm} := \braopket{n}{U(g)}{m}$.

To derive the concrete form of the matrix in general, note that we used above that the irreps of $\SU{2}$ are self-dual, meaning $\dualsector{j} = j$.
%
In general, we need to solve the set of matrix equations $(A^j)^\dagger \pi_j(\ell_i) A^j = \bar{\pi}_{\dualsector{j}}(\ell_i)$ for all generators $\ell_i$ of the real Lie algebra, as the remaining derivations generalizes to any Lie group.
%
Note that depending in the chosen bases, this may or may not translate to an equation like $(A^j)^\dagger \pi_j(j_i) A^j = -\pi_{\dualsector{j}}(j_i)$ with the generators $j_i$ of the complexified algebra.

% =======================================================================================
% =======================================================================================
% =======================================================================================
\section{Fermions}
\label{sec:topo_data:ferm}

\newcommand{\tfermpair}[2]{(#1, #2)}
\newcommand{\fermpair}[2]{\rBr{#1, #2}}

Next, we consider the tensor category $\mathbf{Ferm}$.
%
We can understand it as a description of fermions since the fusion rules and the exchange statistics encoded in the braid give the correct behavior.

We start from the category $\mathbf{FdSHilb}_\Cbb$ of finite dimensional complex super Hilbert spaces.
%
Its objects are pairs $\tfermpair{H}{H'}$ of finite dimensional complex Hilbert spaces, where we can think of $H$ as the ``bosonic" part with even fermionic parity and $H'$ as the ``fermionic" part with odd parity.
%
Its morphisms are pairs $\fermpair{f}{f'}: \fermpair{H}{H'} \to \fermpair{K}{K'}$ of linear maps $f: H \to K$ and $f':H' \to K'$ and composition is componentwise, as is addition of morphisms and multiplication with scalars.
%
The identity morphism is $\id{\fermpair{H}{H'}} = \fermpair{\id{H}}{\id{H'}}$.
%
Direct sums are componentwise direct sums of vector spaces.
%
We obtain the tensor category $\mathbf{Ferm}$ by equipping the super Hilbert spaces with the following monoidal and braiding structures.

For explicit constructions in the following, it is convenient to use the matrix notation in a category with linear structure and direct sums.
%
Let $A = \oplus_{m=1}^M A_m$ be witnessed by inclusions $i_m: A_m \to A$ and projections $p_m: A \to A_m$ and similarly $B = \oplus_{n=1}^N B_n$ by inclusions $\tilde{i}_n: B_n \to B$ and projections $\tilde{p}_n: B \to B_m$.
%
Now for morphisms $f_{m,n} : A_m \to B_n$ we define their matrix as
\begin{equation}
    \begin{pmatrix}
        f_{1,1} & f_{2,1} & \dots & f_{M,1} \\
        f_{1,2} & f_{2,2} & \dots & f_{M,2} \\
        \vdots & \vdots & \ddots & \vdots \\
        f_{1,N} & f_{2,N} & \dots & f_{M,N}
    \end{pmatrix}
    := \sum_{m,n} \tilde{i}_n \compose f_{m,n} \compose p_m
\end{equation}
and conversely, every morphism $g : A \to B$ is equal to the matrix of $g_{m,n} := \tilde{p}_n \compose g \compose i_m$.
%
If the morphisms are linear maps between vector spaces, we can think of this matrix notation as forming the block matrix of the respective matrix representations of the $f_{m,n}$.
%
Composition of matrix morphisms can be carried out similar to matrix-matrix multiplication, i.e.~in short $(f \compose g)_{m,n} = \sum_k f_{m,k} \compose g_{k,n}$.


The tensor product is defined on objects as
\begin{equation}
    \label{eq:topo_data:ferm:def_otimes_objects}
    \fermpair{H}{H'} \otimes \fermpair{K}{K'}
    := \fermpair{(H \otimes K) \oplus (H' \otimes K')}{(H \otimes K') \oplus (H' \otimes K)}
\end{equation}
and encodes the fermionic statistics; a composite system of two fermionic degrees of freedom $\fermpair{0}{H'}$ and $\fermpair{0}{K'}$ behaves as a boson $\fermpair{H' \otimes K'}{0}$.
%
The tensor product of morphisms $\fermpair{f}{f'} : \fermpair{H}{H'} \to \fermpair{K}{K'}$ and $\fermpair{g}{g'} : \fermpair{L}{L'} \to \fermpair{M}{M'}$ is defined as
\begin{equation}
    \fermpair{f}{f'} \otimes \fermpair{g}{g'}
    := \fermpair{ 
        \begin{pmatrix} f \otimes g & 0 \\ 0 & f' \otimes g' \end{pmatrix}
    }{
        \begin{pmatrix} f \otimes g' & 0 \\ 0 & f' \otimes g \end{pmatrix}
    }
    ,
\end{equation}
structurally very similar to the tensor product of objects, except diagonal matrices of morphisms take the place of direct sums of spaces.
%
To read explicit forms of morphisms like above, first expand the domain and codomain
of $\fermpair{f}{f'} \otimes \fermpair{g}{g'} : \fermpair{H}{H'} \otimes \fermpair{L}{L'} \to \fermpair{K}{K'} \otimes \fermpair{M}{M'}$ using~\eqref{eq:topo_data:ferm:def_otimes_objects}.
%
Now, each component of such a morphism is a map between the direct sums, resulting from expanding the tensor product, and can be given as a matrix.

The monoidal unit is $I = \fermpair{\Cbb}{ 0}$.
%
The associator is given by
\begin{equation}
    \label{eq:topo_data:ferm:associatior}
    \setlength{\arraycolsep}{0mm}
    \alpha_{\fermpair{A}{A'}\fermpair{B}{B'}\fermpair{C}{C'}}
    = \fermpair{
        \begin{pmatrix}
            \alpha_{ABC} & 0 & 0 & 0 \\
            0 & 0 & \alpha_{AB'C'} & 0 \\
            0 & 0 & 0 & \alpha_{A'BC'} \\
            0 & \alpha_{A'B'C} & 0 & 0
        \end{pmatrix}
    }{
        \begin{pmatrix}
            \alpha_{ABC'} & 0 & 0 & 0 \\
            0 & 0 & \alpha_{AB'C} & 0 \\
            0 & 0 & 0 & \alpha_{A'BC} \\
            0 & \alpha_{A'B'C'} & 0 & 0
        \end{pmatrix}
    }
    ,
\end{equation}
where $\alpha_{ABC} : (A \otimes B) \otimes C \isoTo A \otimes (B \otimes C)$ are the associators of $\mathbf{FdHilb}_\Cbb$.
%
The definition in terms of matrices suppresses some isomorphisms of the form $(H \oplus K) \otimes L \isoTo (H \otimes L) \oplus (K \otimes L)$ to map the domain $(\fermpair{A}{A'} \otimes \fermpair{B}{B'}) \otimes \fermpair{C}{C'}$ such that each component is a flat direct sum and we can apply the matrix notation.

The unitors are
\begin{equation}
    \lambda_{\fermpair{A}{A'}}
    = \fermpair{
        \begin{pmatrix} \lambda_A & 0 \end{pmatrix}
    }{
        \begin{pmatrix} \lambda_{A'} & 0 \end{pmatrix}
    }
    \qquad
    \rho_{\fermpair{A}{A'}}
    = \fermpair{
        \begin{pmatrix} \rho_A & 0 \end{pmatrix}
    }{
        \begin{pmatrix} 0 & \rho_{A'} \end{pmatrix}
    }
    .
\end{equation}

The dagger of morphisms is componentwise $\fermpair{f}{g}^\dagger = \fermpair{f^\dagger}{g^\dagger}$.

The dual is also componentwise, $\dualspace{\fermpair{H}{K}} = \fermpair{\dualspace{H}}{\dualspace{K}}$, where the cup and cap are given by
\begin{equation}
    \eta_{\fermpair{H}{H'}}
    = \fermpair{\begin{pmatrix}\eta_H \\ \eta_{H'}\end{pmatrix}}{0}
    \qquad
    \epsilon_{\fermpair{H}{H'}}
    = \fermpair{\begin{pmatrix} \epsilon_H & \epsilon_{H'}\end{pmatrix}}{0}
    ~.
\end{equation}

The braid is given by
\begin{equation}
    \tau_{\fermpair{A}{A'}, \fermpair{B}{B'}}
    = \fermpair{
        \begin{pmatrix} \tau_{A,B} & 0 \\ 0 & -\tau_{A',B'} \end{pmatrix}
    }{
        \begin{pmatrix} \tau_{A,B'} & 0 \\ 0 & \tau_{A',B} \end{pmatrix}
    }
    ,
\end{equation}
where $\tau_{A,B} : A \otimes B \to B \otimes A, \ket{a} \otimes \ket{b} \mapsto \ket{b} \otimes \ket{a}$ is the braid of $\mathbf{FdHilb}_\Cbb$.
%
We see the fermionic exchange statistics; braiding a fermionic state around a fermionic state gives a minus sign, while all other braidings have a plus sign.
%
The braid is symmetric.
%

This fully defines the category and we now turn to the topological data.
%
Checking the compatibility axioms to verify that the definitions above do indeed yield a tensor category is cumbersome, but straight-forward, as is deriving the topological data below.


There are two sectors, the trivial sector, or boson $I = \fermpair{\Cbb}{0}$ and the fermion $\psi = \fermpair{0}{\Cbb}$, such that $\mathcal{S} = \set{I, \psi} = \Zbb_2$, where we identify $I = 0 \in \Zbb_2$ and $\psi = 1 \in \Zbb_2$ for explicit numerical values below.
%
Both sectors are self-dual; $\dualsector{\mathbf{a}} = \mathbf{a}$.
%
The fusion rules resulting from the tensor product defined above are $I \otimes I \cong I \cong \psi \otimes \psi$ and $I \otimes \psi \cong \psi \cong \psi \otimes I$.
%
This category shares a property with representations of abelian groups, which we call \emph{unique fusion}, namely that the tensor product of two sectors $a, b$ decomposes as only a single sector, such that $N^{ab}_c$ is non-zero for only a single sector $c$, where it is one.
%
We write $a + b$ for that single sector such that $a \otimes b \cong a + b$.
%
We find
\begin{equation}
    N^{ab}_{c} = \delta_{c, a + b}
    .
\end{equation}
%
Since we have unique fusion, the entire fusion channel is determined by the input sectors, i.e.~the upper indices of the F, R, C, B symbols.
%
We therefore only give the values of the symbols with the unique indices that give a consistent fusion channel.

The fusion tensors are
\begin{gather}
    \begin{gathered}
        X^{II}_I = \fermpair{\begin{pmatrix} X \\ 0\end{pmatrix}}{0}
        \qquad X^{I\psi}_\psi = \fermpair{0}{\begin{pmatrix} X \\ 0\end{pmatrix}}
        \\
        X^{\psi I}_\psi = \fermpair{0}{\begin{pmatrix} 0 \\ X\end{pmatrix}}
        \qquad X^{\psi\psi}_I = \fermpair{\begin{pmatrix} 0 \\ X\end{pmatrix}}{0}
        ,
    \end{gathered}
\end{gather}
where $X : \Cbb \otimes \Cbb \to \Cbb , x \otimes y \to xy$ is the fusion tensor of $\mathbf{FdHilb}_\Cbb$.

The Z isomorphism are simply $Z_I = \fermpair{Z}{0}$ and $Z_\psi = \fermpair{0}{Z}$, where $Z : \dualspace{\Cbb} \to \Cbb , z \mapsto z$ is the Z isomorphism of $\mathbf{FdHilb}_\Cbb$.

As for all categories with unique fusion, we find that the F symbol for valid fusion channels is
\begin{equation}
    \big[ F^{abc}_{a + b + c} \big]^{b + c,1,1}_{a + b,1,1} = 1
    .
\end{equation}

The R symbol is given by
\begin{equation}
    \big[R^{ab}_{a + b}\big]^1_1
    = 1 - 2ab
    = \begin{cases} -1 & a = b = \psi \\ 1 & \text{else}\end{cases}
\end{equation}
as expected; if we braid two fermions, we get a minus sign, and a plus sign otherwise.

The resulting expression for the C symbol is 
\begin{equation}
    \big[ C^{abc}_{a + b + c} \big]^{a + b,1,1}_{a + c,1,1} = 1 - 2 bc
    = \begin{cases} -1 & b = c = \psi \\ 1 & \text{else}\end{cases}
\end{equation}

and the B symbol is again trivial
\begin{equation}
    \big[B^{ab}_{a + b}\big]^1_1 = 1
    .
\end{equation}

Both sectors are one-dimensional $d_a = 1$ and have a positive Frobenius indicators $\chi_a = +1$.
%
The twist is $\Theta_a = (-1)^a = 1 - 2a$.
%
In that sense, we have equipped $\mathbf{FdSHilb}_\Cbb$ with a non-trivial twist.

% =======================================================================================
% =======================================================================================
% =======================================================================================
\section{Fibonacci Anyons}
\label{sec:topo_data:fib}

Next, we consider the tensor category $\mathbf{Fib}$.
%
We can understand it as a description of Fibonacci anyon excitations since the fusion rules and the exchange statistics encoded in the braid give the correct behavior.

The construction is similar to the category $\mathbf{Ferm}$ of fermions and comments there apply.
%
We start from the same underlying objects, tuples $(H,H')$ of finite dimensional Hilbert spaces, and morphisms, tuples $(f,f'): (H,H') \to (K,K')$ of linear maps $f: H \to K$ and $f': H' \to K'$.
%
Again, composition, addition, scalar multiplication, identities, direct sums and the dagger are all componentwise.

We choose a different tensor product for $\mathbf{Fib}$, however, namely
\begin{equation}
    \label{eq:topo_data:fib:def_otimes_objects}
    \fermpair{H}{H'} \otimes \fermpair{K}{K'}
    := \fermpair{(H \otimes K) \oplus (H' \otimes K')}{(H \otimes K') \oplus (H' \otimes K) \oplus (H' \otimes K')}
    .
\end{equation}
%
Note the additional term in the second component compared to ~\eqref{eq:topo_data:ferm:def_otimes_objects}.
%
The monoidal unit is $I=\fermpair{\Cbb}{0}$.
%
The associator $$\alpha_{\fermpair{A}{A'}\fermpair{B}{B'}\fermpair{C}{C'}} = \fermpair{\alpha^I_{\fermpair{A}{A'}\fermpair{B}{B'}\fermpair{C}{C'}}}{\alpha^\tau_{\fermpair{A}{A'}\fermpair{B}{B'}\fermpair{C}{C'}}}$$ is a tuple of the following two components

\begin{gather}
\begin{gathered}
    \alpha^I_{\fermpair{A}{A'}\fermpair{B}{B'}\fermpair{C}{C'}}
    =
    \begin{pmatrix}
        \alpha_{ABC} & 0 & 0 & 0 & 0 \\
        0 & 0 & \alpha_{AB'C'} & 0 & 0 \\
        0 & 0 & 0 & \alpha_{A'BC'} & 0 \\
        0 & \alpha_{A'B'C} & 0 & 0 & 0 \\
        0 & 0 & 0 & 0 & \alpha_{A'B'C'}
    \end{pmatrix}
    \\
    \setlength{\arraycolsep}{0pt}
    \alpha^\tau_{\fermpair{A}{A'}\fermpair{B}{B'}\fermpair{C}{C'}}
    =
    \begin{pmatrix}
        \alpha_{ABC'} & 0 & 0 & 0 & 0 & 0 & 0 & 0 \\
        0 & 0 & \alpha_{AB'C} & 0 & 0 & 0 & 0 & 0 \\
        0 & 0 & 0 & 0 & 0 & \alpha_{AB'C'} & 0 & 0 \\
        0 & 0 & 0 & \alpha_{A'BC} & 0 & 0 & 0 & 0 \\
        0 & \tfrac{1}{\phi} \alpha_{A'B'C'} & 0 & 0 & 0 & 0 & 0 & \tfrac{1}{\sqrt{\phi}} \alpha_{A'B'C'} \\
        0 & 0 & 0 & 0 & 0 & 0 & \alpha_{A'BC'} & 0 \\
        0 & 0 & 0 & 0 & \alpha_{A'B'C} & 0 & 0 & 0 \\
        0 & \tfrac{1}{\sqrt{\phi}} \alpha_{A'B'C'} & 0 & 0 & 0 & 0 & 0 & -\tfrac{1}{\phi} \alpha_{A'B'C'} \\
    \end{pmatrix}
    ~,
\end{gathered}
\end{gather}
where $\phi = (1 + \sqrt{5}) / 2$ is the golden ratio and we suppress isomorphisms similar to~\eqref{eq:topo_data:ferm:associatior}.

The unitors are
\begin{equation}
    \lambda_{\fermpair{A}{A'}}
    = \fermpair{
        \begin{pmatrix} \lambda_A & 0 \end{pmatrix}
    }{
        \begin{pmatrix} \lambda_{A'} & 0 & 0 \end{pmatrix}
    }
    \qquad
    \rho_{\fermpair{A}{A'}}
    = \fermpair{
        \begin{pmatrix} \rho_A & 0 \end{pmatrix}
    }{
        \begin{pmatrix} 0 & \rho_{A'} & 0\end{pmatrix}
    }
    .
\end{equation}

The dual object is componentwise $\dualspace{\fermpair{H}{K}} = \fermpair{\dualspace{H}}{\dualspace{K}}$, where the cup and cap are given by
\begin{equation}
    \eta_{\fermpair{H}{H'}}
    = \fermpair{\begin{pmatrix}\eta_H \\ \eta_{H'}\end{pmatrix}}{0}
    \qquad
    \epsilon_{\fermpair{H}{H'}}
    = \fermpair{\begin{pmatrix} \epsilon_H & \epsilon_{H'}\end{pmatrix}}{0}
\end{equation}
which is structurally very similar to fermions, but note that the zero maps in the second components have a different type.

The braid is given by
\begin{equation}
    \tau_{\fermpair{A}{A'}, \fermpair{B}{B'}}
    = \fermpair{
        \begin{pmatrix} \tau_{A,B} & 0 \\ 0 & \eto{-\tfrac{4}{5}\pi\im} \tau_{A',B'} \end{pmatrix}
    }{
        \begin{pmatrix} \tau_{A,B'} & 0 & 0\\ 0 & \tau_{A',B} & 0 \\ 0 & 0 & \eto{\tfrac{3}{5}\pi\im} \tau_{A',B'} \end{pmatrix}
    }
\end{equation}
Here, we can already heuristically interpret the exchange statistics.
%
Braiding two $\tau$'s results in a phase that depends on the joint state or ``fusion channel", it is $\eto{-\tfrac{4}{5}\pi\im} \tau_{A',B'}$ if they fuse to $I$ and $\eto{\tfrac{3}{5}\pi\im}$ if they fuse to $\tau$.
%
In particular, the braid is \emph{not} symmetric.

Now for the topological data;
%
The sectors are the trivial sector or vacuum $I = \fermpair{\Cbb}{0}$ and the tau anyon $\tau = \fermpair{0}{\Cbb}$, such that $\mathcal{S} = \set{I, \tau}$.
%
They are both self-dual $\dualsector{a} = a$ and the fusion rules
\begin{equation}
    I \otimes I \cong I
    \quad 
    I \otimes \tau \cong \tau
    \quad 
    \tau \otimes I \cong \tau
    \quad 
    \tau \otimes \tau \cong I \oplus \tau
\end{equation}
are indeed the correct fusion rules for Fibonacci anyons and give rise to the N symbol
\begin{equation}
    N^{Ib}_c = \delta_{b,c} \qquad N^{aI}_c = \delta_{a,c} \qquad N^{\tau\tau}_c = 1
    .
\end{equation}
Note that the N symbol only takes on values $N^{ab}_c \in \set{0, 1}$, such that all multiplicity labels for valid fusion channels are $\mu=1$.

The fusion tensors are
\begin{gather}
\begin{gathered}
    X^{II}_I = \fermpair{\begin{pmatrix} X \\ 0\end{pmatrix}}{0}
    \qquad X^{I\tau}_\tau = \fermpair{0}{\begin{pmatrix} X \\ 0 \\ 0\end{pmatrix}}
    \qquad X^{\tau I}_\tau = \fermpair{0}{\begin{pmatrix} 0 \\ X \\ 0\end{pmatrix}}
    \\
    \qquad X^{\tau\tau}_I = \fermpair{\begin{pmatrix} 0 \\ X\end{pmatrix}}{0}
    \qquad X^{\tau\tau}_\tau = \fermpair{0}{\begin{pmatrix} 0 \\ 0 \\ X\end{pmatrix}}
    .
\end{gathered}
\end{gather}
The Z isomorphism are simply $Z_I = \fermpair{Z}{0}$ and $Z_\tau = \fermpair{0}{Z}$.

The only non-trivial F symbol is
\begin{equation}
    \big[ F^{\tau\tau\tau}_{\tau} \big]^{e,1,1}_{f,1,1}
    = \begin{cases}
        \phi^{-1} & e = f = I \\
        -\phi^{-1} & e = f = \tau \\
        \phi^{-1/2} & (e, f) \in \set{(I, \tau), (\tau, I)}
    \end{cases}
\end{equation}
and for all other $(a,b,c,d) \neq (\tau,\tau,\tau,\tau)$ we have $[F^{abc}_{d}]^{e,1,1}_{f,1,1} = 1$ where there is only one valid choice for $e,f$ in each case.

The only non-trivial R symbol is
\begin{equation}
    \big[R^{\tau\tau}_{c}\big]^1_1
    = \begin{cases}
        \eto{-\tfrac{4}{5}\pi\im } & c = I \\
        \eto{\tfrac{3}{5}\pi\im } & c = \tau
    \end{cases}
\end{equation}
and for all other $(a,b) \neq (\tau,\tau)$ we have $[R^{ab}_c]^1_1 = 1$ and there is only one valid choice for $c$ in each case.

The non-trivial B symbols are $[B^{\tau\tau}_I]^1_1 = \phi^{-1/2}$ and $[B^{I\tau}_\tau]^1_1 = \phi^{1/2}$ and all others are $[B^{\tau\tau}_\tau]^1_1 = [B^{\tau I}_\tau]^1_1 = [B^{II}_I]^1_1 = 1$.

The quantum dimensions are $d_I = 1$ and $d_\tau = \phi$, the Frobenius Schur indicators are all positive $\chi_a = +1$ and the twists are $\theta_I = 1$ and $\theta_\tau = \phi^{-1} \eto{-\tfrac{4}{5}\pi\im } + \eto{\tfrac{3}{5}\pi\im }$.


% =======================================================================================
% =======================================================================================
% =======================================================================================
\section{Combining symmetries}
\label{sec:topo_data:product}

Combining multiple symmetries is formalized by Deligne's tensor product.
%
For tensor categories $\mathbf{A}$ and $\mathbf{B}$, a Deligne tensor product is a category $\mathbf{A} \boxtimes \mathbf{B}$ with a functor $\boxtimes: \mathbf{A} \times \mathbf{B} \to \mathbf{A} \boxtimes \mathbf{B}$ that has the following defining property.
%
Firstly, $\boxtimes$ is right-exact in both arguments and, secondly, for any other tensor category $\mathbf{C}$ and functor $F: \mathbf{A} \times \mathbf{B} \to \mathbf{C}$ that is right exact in both arguments there is a unique right exact functor $\tilde{F}: \mathbf{A} \boxtimes \mathbf{B} \to \mathbf{C}$ such that $F = \tilde{F} \compose \boxtimes$.
%
Under our assumptions for a tensor category, such a product is guaranteed to exist and is unique up to a unique equivalence~\cite[Prop 1.11.2]{etingof2015} and can be equipped with all the structures of a tensor category in a canonical way~\cite[Prop 4.6.1]{etingof2015}.
%
Dealing with it explicitly and deriving the following statements rigorously is beyond the scope of this thesis.
%
We expect that the explicit construction in~\cite[Def. 8.45]{heunen2019} may prove useful.
%
The following statements in this section are thus to be taken as conjecture.



We expect that the Deligne tensor product of tensor categories that describe individual symmetries describes the composite symmetry in the following senses;
%
For symmetry groups $G$ and $H$ the Deligne product can be chosen as $\mathbf{FdRep}_{G} \boxtimes \mathbf{FdRep}_{H} = \mathbf{FdRep}_{G\times H}$ and thus describes the symmetry group $G \times H$.
%
For a symmetry group $G$, we conjecture that the Deligne product $\mathbf{Ferm}_G := \mathbf{Ferm} \boxtimes \mathbf{FdRep}_{G}$ is a category that can be explicitly defined by generalizing the definition of $\mathbf{Ferm}$ from tuples of Hilbert spaces (objects in $\mathbf{FdHilb}_\Cbb$) to tuples of representations of $G$, that is objects of $\mathbf{FdRep}_{G}$.
%
It thus represents fermionic degrees of freedom with a symmetry group.
%
Similarly for $\mathbf{Fib}$, we expect $\mathbf{Fib}_G := \mathbf{Fib} \boxtimes \mathbf{FdRep}_{G}$ to describe Fibonacci anyon excitations with a symmetry group, with an analogous generalization of the definition of $\mathbf{Fib}$.
%
We expect that similar constructions apply to other modular anyon categories, other than $\mathbf{Fib}$.

Let us now conjecture the topological data.
%
The sectors are given by Deligne tensor products of sectors, that is 
\begin{equation}
    \mathcal{S}_{A \boxtimes B} = \setdef{a \boxtimes b}{a\in\mathcal{S}_A, b\in\mathcal{S}_B}
\end{equation}
which we can just understand as tuples $(a,b)$ of respective sectors in the practical implementation.
%
The topological data mostly just factorizes;
%
The N symbol is given by 
\begin{equation}
    N^{a_1 \boxtimes a_2, b_1 \boxtimes b_2}_{c_1\boxtimes c_2} = N^{a_1,b_1}_{c_1} N^{a_2,b_2}_{c_2}
\end{equation}
such that multiplicity labels $\mu = 1, \dots, N^{a_1 \boxtimes a_2, b_1 \boxtimes b_2}_{c_1\boxtimes c_2}$ can be recast as tuples $(\mu_1, \mu_2)$ of separate multiplicity labels $\mu_i = 1,\dots N^{a_i,b_i}_{c_i}$, i.e.~by using strides.
%
The fusion tensors are
\begin{equation}
    X^{a_1 \boxtimes a_2, b_1 \boxtimes b_2}_{c_1\boxtimes c_2,(\mu_1,\mu_2)}
    = X^{a_1,b_1}_{c_1,\mu_1} \boxtimes X^{a_2,b_2}_{c_2,\mu_2}
\end{equation}
and the Z isomorphisms are
\begin{equation}
    Z_{a_1 \boxtimes a_2} = Z_{a_1} \boxtimes Z_{a_2}
    ~.
\end{equation}
%
The remaining topological data all factorizes, such as e.g. for the R symbol
\begin{equation}
    \big[ R^{a_1 \boxtimes a_2, b_1 \boxtimes b_2}_{c_1\boxtimes c_2} \big]^{(\mu_1, \mu_2)}_{(\nu_1, \nu_2)}
    = \big[ R^{a_1,b_1}_{c_1} \big]^{\mu_1}_{\nu_1} \big[ R^{a_2,b_2}_{c_2} \big]^{\mu_2}_{\nu_2} 
\end{equation}
i.e.~as a matrix of multiplicity indices, it is the Kronecker matrix product of the respective symbols from the separate categories.
%
Similar expressions hold for the F, C and B symbols.
%
The quantum dimensions, Frobenius Schur indicator and twist all factorize too, e.g.~$d_{a_1\boxtimes a_2} = d_{a_1} d_{a_2}$.

Generalizations to combinations of more than two symmetries is straight-forward.