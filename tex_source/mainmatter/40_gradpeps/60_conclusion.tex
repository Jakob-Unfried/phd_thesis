We have proposed an approach to the gradient-based optimization of tensor network states, in particular finite \acro{peps}, in light of approximate contractions.
%
This avoids the pathological states that can result from naively optimizing an approximately contracted cost function using automatic differentiation.
%
By instead evaluating the exact expression of the gradient of the cost function using the same approximate contraction method, we have achieved a stable and successful optimization algorithm.
%
In this proof of principle work, we do not exploit symmetries, limiting the bond dimension achievable on larger systems.
%
Nevertheless, we see better results than any other publication (for \emph{finite} \acro{peps}) we are aware of at the lowest bond dimension $D=2$, where we reach convergence and achieve comparable energies at the larger bond dimensions.
%
Notably, our ground state results, simulated sequentially on tens of \acro{cpu} cores, yield results competitive with the massively parallel \acro{dmrg} simulation of Ref.~\cite{ganahl2023} running on a thousand cores of specialized \acro{tpu} hardware.
%
This discrepancy further illustrates the potential of natively two-dimensional tensor network methods, such as \acro{peps}, for the simulation of 2D quantum systems.
%
The gradient-based approach allowed us to perform dynamics simulations even at the lowest bond dimensions, which -- to our knowledge -- do not admit accurate time evolution by other means.



Our results showcase conventional wisdom established when comparing \acrofull{su} and \acrofull{fu} ground state searches that the maximal bond dimension reached by a \acro{peps} simulation only loosely correlates with the quality of the results.
%
We observe that local updates do not fully exhaust the variational power of the fixed bond dimension manifold.
%
The global, gradient based method on finite system offers an approach complementary to the \acro{tebd}-style local updates on large unit cell infinite systems of Ref.~\cite{espinoza2024}, to extract spectral functions in 2D.
%
Offering a competitive alternative to \acro{mps} simulations on cylinder geometries, especially when considering more strongly correlated models than the \acro{tfim}, is an open challenge.



Nevertheless, gradient-based approaches seem to be a promising avenue to leverage the full variational power of the \acro{peps} ansatz, depending on how far performance can be pushed.
%
In any case, hybrid approaches may prove fruitful, leveraging the performance of established algorithms, such as e.g.~\acro{fu} or similar, to get a good approximation, then further improving on it with a few costly, but effective gradient steps, as is done in Ref.~\cite{scheb2023}.
%
The different convergence properties of global energy minimization and imaginary time evolution may also complement each other to avoid local minima and plateaus.


The future directions are to push the performance to fully converge the ground state results for the larger systems and larger bond dimensions, and we refer future readers to a future published version of~\cite{unfried2024}.
%
We also plan to study the implications of the simplification of the \acro{autodiff} formula of the truncated \acro{svd}, given in section~\ref{subsec:gradpeps:autodiff:trunc_svd} for gradient-based optimization using gradients from \acro{autodiff}, and if it might stabilize the optimization.
%
It may also prove beneficial for performance to employ the \acro{gpu}-friendly truncation routines discussed in chapter~\ref{ch:truncation} in the \acro{bmps} contraction to exploit hardware acceleration.
