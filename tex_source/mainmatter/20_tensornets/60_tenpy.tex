The \acrofull{tenpy} library~\cite{tenpySoftware} is a python package for tensor network simulations.
%
The library is based on previous (non-public) codes used in references~\cite{kjall2013, zaletel2013}, developed by the authors of these works.
%
It was rewritten in its current structure by Johannes Hauschild during his PhD and released as an open source package~\cite{hauschild2018b}.
%
The package is currently maintained by Johannes Hauschild and the author of this thesis, and has recently seen a version 1.0 release~\cite{hauschild2024}.
%
Current active development aims to improve the low-level part of the package handling linear algebra of tensors, to support nonabelian symmetries, generalized symmetries, and fermionic or anyonic degrees of freedom -- see chapter~\ref{ch:nonabelian} -- as well as hardware acceleration on \acrop{gpu}, to incorporate e.g.~the concepts discussed in chapter~\ref{ch:truncation}.
%
A prototype of the new implementation was developed by the author of this thesis, and is currently being optimized for performance and incorporated into the rest of the library.


The package offers high-level functionality that abstracts entire simulations of, e.g.~a response function from time evolution or a phase diagram sweep that performs ground state search at many parameter values and extracts order parameters.
%
These are implemented in terms of ``mid-level" algorithms, such as e.g.~the \acro{dmrg}, \acro{tebd} and \acro{mpoEvolution} algorithms discussed in sections~\ref{subsec:mps:tebd}-\ref{subsec:mps:mpo_evolution} respectively, as well as \acro{tdvp} and \acro{vumps}.
%
The focus is currently mostly on \acro{mps} simulations in one and two spatial dimensions.
%
Many features exist to facilitate the specification of the physical models, in terms of a 1D, quasi-1D, or 2D lattice geometry.
%
As a result, Hamiltonians can be specified in the natural language of local operators and couplings on the lattice, and the construction of e.g.~\acro{mpo} representations of the Hamiltonian is fully automated.
%
This allows non-experts to set up, e.g.~a \acro{dmrg} simulation for a particular physical model without knowing any details of the tensor network representation of its Hamiltonian, or run \acro{tebd} without interacting with the Trotterization and the order in which gates are applied.
%
The low-level functionality offers linear algebra routines on symmetric tensors, supporting abelian symmetry groups, as discussed in section~\ref{sec:tensornets:symmetries}.

%
The code is open source and maintained publicly on GitHub~\cite{tenpySoftware}.
%
Refer to the online documentation\footnote{
    \url{https://tenpy.readthedocs.io/en/latest/}
}
to get started.
